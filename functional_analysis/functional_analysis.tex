\documentclass[]{article}
\usepackage{amsmath}
\usepackage[bitstream-charter]{mathdesign}
\usepackage[dvipsnames]{xcolor}
\usepackage{hyperref}
\hypersetup{
  pdftitle={Functional Analysis},
  pdfauthor={Kexing Ying},
  colorlinks=true,
  linkcolor=Maroon,
  filecolor=Maroon,
  citecolor=Maroon,
  urlcolor=Maroon,
  pdfcreator={LaTeX via pandoc}}

\usepackage[margin = 1.5in]{geometry}
\usepackage{graphicx}
\makeatletter
\def\maxwidth{\ifdim\Gin@nat@width>\linewidth\linewidth\else\Gin@nat@width\fi}
\def\maxheight{\ifdim\Gin@nat@height>\textheight\textheight\else\Gin@nat@height\fi}
\makeatother
% Scale images if necessary, so that they will not overflow the page
% margins by default, and it is still possible to overwrite the defaults
% using explicit options in \includegraphics[width, height, ...]{}
\setkeys{Gin}{width=\maxwidth,height=\maxheight,keepaspectratio}

\setlength{\emergencystretch}{3em} % prevent overfull lines
\providecommand{\tightlist}{%
  \setlength{\itemsep}{0pt}\setlength{\parskip}{0pt}}
\setcounter{secnumdepth}{5}

\setlength\parindent{0pt} % no indentation for paragraphs
\setlength{\parskip}{3pt} % add a little space between paragraphs

\usepackage{tikz}
\usepackage{physics}
\usepackage{amsthm}
\usepackage{mathtools}

\theoremstyle{definition}
\newtheorem{theorem}{Theorem}
\newtheorem{definition}{Definition}[section]
\newtheorem{lemma}{Lemma}[section]
\newtheorem{corollary}{Corollary}[theorem]
\newtheorem{proposition}{Proposition}[section]
\newtheorem{example}{Example}[section]
\newtheorem*{remark}{Remark}

\title{Functional Analysis}
\author{Kexing Ying}

\begin{document}
\maketitle
\begin{abstract}
  \noindent This note contains parts of the course \textit{Functional Analysis} taught by András Zsák for 
  Part~III students at the University of Cambridge. I will omit the initial parts of the course 
  reviewing linear operator theory and numerous Hahn-Banach theorems. I will also omit the proof of the 
  Radon-Nikodym theorem. 
\end{abstract}

\tableofcontents

\newpage
\section{The Dual of \texorpdfstring{\(L_p\)}{Lp} and \texorpdfstring{\(C(K)\)}{C(K)}}

We will always work in the measure space \((\Omega, \mathcal{F}, \mu)\). We recall the Radon-Nikodym 
theorem and its related results.

\begin{theorem}[Hahn decomposition]
  Given a signed measure \(\nu : \mathcal{F} \to \mathbb{R}\), there exists a disjoint partition 
  \(A, B \in \mathcal{F}\), \(A \sqcup B = \Omega\) such that for all \(S \subseteq A\), 
  \(\nu(S) \geq 0\) and for all \(S \subseteq B\), \(\nu(S) \leq 0\).
\end{theorem}

\begin{corollary}[Hahn-Jordan decomposition of a signed measure]
  Given a signed measure \(\nu\), there exists unique measures \(\nu^+, \nu^-\) such that for all 
  \(S \in \mathcal{F}\), \(\nu(S) = \nu^+(S) - \nu^-(S)\).
\end{corollary}

\begin{theorem}[Radon-Nikodym]
  Given \(\mu\) is \(\sigma\)-finite, \(\nu : \mathcal{F} \to \mathbb{C}\) is a complex measure 
  such that \(\nu \ll \mu\), there exists a unique \(f \in L_1(\mu)\) such that for all 
  \(S \in \mathcal{F}\), 
  \[\nu(A) = \int_A f \dd \mu.\]
\end{theorem}

\begin{remark}
  This \(f\) is said to be the Radon-Nikodym derivative of \(\nu\) with respect to \(\mu\) and we 
  denote it by \(\dv*{\nu}{\mu}\).
\end{remark}

\begin{remark}
  In the case \(\nu\) is not necessarily absolutely continuous with respect to \(\mu\), we can  
  decompose \(\nu = \nu_1 + \nu_2\) where \(\nu_1, \nu_2\) are complex measures such that 
  \(\nu_1 \ll \mu\) and \(\nu_2 \perp \mu\) (i.e. there exists some \(S \in \mathcal{F}\) such that 
  \(\nu_2(S) = 0 = \mu(S^c)\)).
\end{remark}

\subsection{Dual of \texorpdfstring{\(L_p\)}{Lp}}

Utilizing the Radon-Nikodym theorem, we in this section show that for all \(p \in [1, \infty)\), 
\(L_p^*\) is isometrically isomorphic to \(L_q\) for \(p, q\) Hölder conjugates.

The map we consider is 
\[\phi : L_q \to L_p^* : g \mapsto \phi_g\]
where we define \(\phi_g(f) := \int fg \dd \mu\). This map is well defined since 
\(|\phi_g(f)| \le \|g\|_q\|f\|_p\) and so \(\|\phi_g\| \le \|g\|_q < \infty\). As \(\phi\) is 
clearly linear, this furthermore shows that \(\phi\) is bounded.

\begin{theorem}
  For \(p \in (1, \infty)\), \(\phi\) is a isometric isomorphism between \(L_q\) and \(L_p^*\).
  Furthermore, in the case that \(\mu\) is \(\sigma\)-finite, the same remains to hold for \(p = 1\).
\end{theorem}
\begin{proof}
  We first consider the case \(p \in (1, \infty)\) and we show that \(\phi\) is isometric. 
  
  Let \(g \in L_q\), we have already shown that \(\|\phi_g\| \le \|g\|_q\). We now show the converse 
  inequality. Define 
  \[f = \begin{cases}
    \frac{|g|^q}{g}, & g \neq 0,\\
    0, & g = 0.
  \end{cases}\]
  It suffices to show \(|\phi_g(f)| / \|f\|_p\) achieves \(\|g\|_q\). Indeed,
  \[\int |f^p| \dd \mu = \int |g|^{(q - 1)p} \dd \mu = \int |g|^q \dd \mu < \infty\]
  and so \(f \in L_p\) and \(\|f\|_p^p = \|g\|_q^q\). Thus, 
  \[|\phi_g(f)| = \int |g|^q \dd\mu = \|g\|_q^q = \|f\|_p^p,\]
  implying 
  \[\frac{|\phi_g(f)|}{\|f\|_p} = \|f\|_p^{p - 1} = \|g\|_q^{\frac{q(p - 1)}{p}} = \|g\|_q\]
  as claimed.

  We now show that \(\phi\) is surjective. We first consider the case that \(\mu\) is finite.

  Fix \(\psi \in L_p^*\). Define 
  \[\nu(A) = \psi(\mathbf{1}_A).\]
  I claim that \(\nu\) is a complex measure. Indeed, 
  \begin{itemize}
    \item \(\nu(\varnothing) = \psi(0) = 0\), and
    \item for \((A_n) \subseteq \mathcal{F}\) disjoint,
    \begin{align*}
      \left|\nu\left(\bigcup_n A_n\right) - \sum_{n = 1}^N \nu(A_n)\right| 
      & = \left|\psi\left(\mathbf{1}_{\bigcup_n A_n} - \sum_{n = 1}^N \mathbf{1}_{A_n}\right)\right|\\
      & \le \|\psi\| \left\|\mathbf{1}_{\bigcup_n A_n} - \sum_{n = 1}^N \mathbf{1}_{A_n}\right\|_p
        = \|\psi\| \mu\left(\bigcup_{n = N}^\infty A_n\right)^{1 / p}
    \end{align*}
      which converges to \(0\) as \(N \to \infty\) implying \(\sigma\)-additivity.
  \end{itemize}
  Furthermore, it is clear that \(\nu \ll \mu\) and so by the Radon-Nikodym theorem, there exists 
  a unique \(g \in L_1(\mu)\) such that for all \(S \in \mathcal{F}\), 
  \[\psi(\mathbf{1}_S) = \nu(S) = \int_S g \dd \mu.\]
  Thus, it follows that for any simple function \(f\), \(\int fg \dd \mu = \psi(f)\). 

  Now approximating \(f \in L_\infty\) by simple functions \(f_n \uparrow f\), we have 
  \[\psi(f) = \lim_{n \to \infty}\psi(f_n) 
    = \lim_{n \to \infty} \int f_n g \dd \mu \stackrel{\text{MCT}}{=} \int fg \dd \mu.\]
  With this, we would like to conclude by the density of \(L_\infty\) in \(L_p\) (this is what requires 
  \(\mu\) to be finite). However to do so, we need to first check that \(g \in L_q\) and so \(\phi_g\) 
  is in fact in \(L_p^\infty\). Let us check this now: 

  For \(n \in \mathbb{N}\), let \(A_n = \{0 < |g| < n\}\) and 
  \(f = \mathbf{1}_{A_n}|g|^q / g \in L_\infty \subseteq L_p\). 
  Then, \[\int |g|^q \dd \mu = \int fg \dd \mu = \psi(f) \le \|\psi\| \|f\|_p = 
    \|\psi\| \left(\int_{A_n} |g|^q \dd \mu\right)^{1 / p}.\] 
  Thus, taking \(n \to \infty\), we have by the monotone convergence theorem 
  \[\|\psi\| \ge \left(\int_{A_n} |g|^q \dd \mu\right)^{1 - 1 / p} = \|g\|_q\]
  and so \(g \in L_q\) as required. With this, by the previous remark, we conclude that \(\phi_g = \psi\) 
  for \(\mu\) finite case by leveraging the density of \(L_\infty\) in \(L_p\).

  Before proving the general case, let us first introduce the following notations: given \(B \in \mathcal{F}\),
  we denote \(\mathcal{F}_B = \{A \in \mathcal{F} \mid A \subseteq B\}\) and \(\mu_B = \mu|_B\) so 
  \((\Omega, \mathcal{F}_B, \mu_B)\) is a measure space and \(L_p(\mu_B) \subseteq L_p(\mu)\). 
  Furthermore, given \(\psi \in L_p(\mu)^*\), we denote \(\psi_B = \psi|_{L_p(\mu_B)}\) so 
  \(\psi_B \in L_p(\mu_B)^*\) and \(\|\psi_B\| \le \|\psi\|\). We note the following claim:

  \textit{Given \(B, C \in \mathcal{F}\) are disjoint, \(\|\psi_{B \cup C}\|^q = \|\psi_B\|^q + \|\psi_C\|^q\).}
  \begin{proof}[Proof of claim]
    Let \(f \in L_p(\mu_{B \cup C})\). Then,
    \begin{align*}
      |\psi_{B \cup C}(f)| & = |\psi_B(f|_B) + \psi_C(f|_C)|\\
      & \le |\psi_B(f|_B)| + |\psi_C(f|_C)|\\
      & \le \|\psi_B\|\|f|_B\|_p + \|\psi_C\|\|f|_C\|_p\\
      & \stackrel{\text{Hölder}}{\le} (\|\psi_B\|^q + \|\psi_C\|^q)^{1 / q}(\|f|_B\|_p^p + \|f|_C\|_p^p)^{1 / p}\\
      & = (\|\psi_B\|^q + \|\psi_C\|^q)^{1 / q}\|f\|_p
    \end{align*}
    implying \(\|\psi_{B \cup C}\|^q \le \|\psi_B\|^q + \|\psi_C\|^q\).

    On the other hand, for the reverse direction, fix \(a, b \ge 0\) such that \(a^p + b^q = 1\) and 
    \[a\|\psi_B\| + b\|\psi_C\| = (\|\psi_B\|^a + \|\psi_C\|^b)^{1 / q}.\]
    Then, given \(f \in L_p(\mu_B), g \in L_p(\mu_C)\), with \(\|f\|_p, \|g\|_p \le 1\), 
    \(\alpha, \beta\) scalars such that \(|\alpha| = |\beta| = 1\) and 
    \[\alpha \psi_B(f) = |\psi_B(f)| \text{ and } \beta\psi_C(g) = |\psi_C(g)|,\] 
    we observe
    \[a |\psi_B(f)| + b|\psi_C(g)| = \psi_{B \cup C}(a\alpha f + b \beta g) \le 
      \|\psi_{B \cup C}\| \|a \alpha f + b \beta g\|_p \le \|\psi_{B \cup C}\|\]
    implying \(\|\psi_{B \cup C}\|^q \ge \|\psi_B\|^q + \|\psi_C\|^q\) as required.
  \end{proof}

  Let us now consider the case \(\mu\) is \(\sigma\)-finite. In this case, by definition, there exists 
  a countable measurable partition of \(\Omega\): \((A_n)\) such that \(\mu(A_n) < \infty\) for all \(n\). 
  So, for \(\psi \in L_p(\mu)^*\), we can restrict \(\psi\) onto \(A_n\) and apply the previous 
  case. Namely, for all \(n\), there exists some \(g_n \in L_q(\mu_{A_n})\) such that
  \[\psi_{A_n}(f) = \int f g_n \dd \mu_{A_n} = \int_{A_n} f g_n \dd\mu.\]
  Observe that for all \(N \in \mathbb{N}\),
  \[\sum_{n = 1}^N \|g_n\|_q^q = \sum_n^N \|\psi_{A_n}\| \stackrel{(*)}{=} 
  \|\psi_{\bigcup_{n = 1}^N A_n}\| \le \|\psi\| < \infty.\]
  where \((*)\) follows by the claim.

  So, by defining \(g = g_n\) on \(A_n\), we have by the monotone convergence theorem 
  \(g \in L_q(\mu)\) and \(\phi_g = \psi\) on \(L_p(\mu_{A_n})\) for all \(n\). Hence, as 
  \(\bigcup L_p(\mu_{A_n})\) has dense linear span, \(\psi = \phi_g\) as required.

  Finally, for the general case, take \(\psi \in L_p(\mu)^*\) and choose \((f_n)\) 
  to be a sequence in \(L_p(\mu)\) such that \(\|f_n\| \le 1\) for all \(n\) and 
  \[\psi(f_n) \to \|\psi\| \text{ as } n \to \infty.\]
  Recall that for \(f \in L_p(\mu)\), 
  \[\{f \neq 0\} = \bigcup_n \{|f| > n^{-1}\}\]
  which is \(\sigma\)-finite as by Markov's inequality,
  \[\mu(\{|f| > n^{-1}\}) \le n^p \|f\|_p^p < \infty\]
  for all \(n\). Thus, defining \(B = \bigcup_n \{f_n \neq 0\}\), \(B\) is \(\sigma\)-finite and 
  by the \(\sigma\)-finite case, there exists some \(g \in L_q(\mu_B)\) such that 
  \(\phi_g = \psi_B\). Now, by the claim,
  \[\|\psi\|^q = \|\psi_B\|^q + \|\psi_{\Omega \setminus B}\|^q\]
  while by construction, \(\|\psi\|^q = \|\psi_B\|^q\). Thus, \(\psi_{\Omega \setminus B} = 0\) 
  and \(\psi = \phi_g\).

  We now start proving the case \(p = \infty\) and \(\mu\) is \(\sigma\)-finite. We first show 
  \(\phi\) is isometric. Let \(g \in L_\infty(\mu)\). We've already shown \(\|\phi_g\| \le \|g\|_\infty\)
  so it suffices to show the reverse inequality. WLOG. assume that \(g \neq 0\) and fix \(0 < s < \|g\|_\infty\) 
  and define \(A = \{|g| > s\}\). Straightaway, we note \(\mu(A) > 0\) and so, as \(\mu\) is \(\sigma\)-finite, 
  there exists some \(B \subseteq A\), \(0 < \mu(B) < \infty\).
  Defining \(f = \mathbf{1}_B|g| / g\), we have \(f \in L_1\) and 
  \[s \le \int_B |g| \dd \mu = \phi_g(f) \le \|\phi_g\| \|f\|_1 = \|\phi_g\|\mu(B).\]
  Hence, \(s \le \|\phi_g\|\) and as \(s < \|g\|_\infty\) was arbitrary, we have \(\|\phi_g\| \ge \|g\|_\infty\) 
  as required.

  For subjectivity, we proceed similarly to the first case. Given \(\psi \in L_1^*\), define 
  \[\nu(A) = \psi(\mathbf{1}_A), \text{ for all } A \in \mathcal{F}.\]
  \(\nu\) is a complex measure and by Radon-Nikodym, there exists some \(g \in L_1\), \(\nu(A) = \int_A g \dd\mu\)
  for all \(A \in \mathcal{F}\). Then, by approximating with simple functions, it is clear that 
  \(\psi(f) = \int fg \dd \mu\) for all \(f \in L_\infty\).

  We now show \(g \in L_\infty\). Fix 
  \[t > \|\psi\|, A = \{|g| > t\}, f = \mathbf{1}_A \frac{|g|}{g}.\]
  Then \(f \in L_\infty\) and thus,
  \[t \mu(A) \le \int_A |g| \dd \mu = \int fg \dd \mu = \psi(f) \le \|\psi\|\|f\|_1 = \|\psi\| \mu(A).\]
  However, as \(t > \|\psi\|\) by definition, we have \(\mu(A) = 0\) implying \(g \in L_\infty\).

  So far we've shown \(\psi = \phi_g\) on \(L_\infty\). To show \(\psi = \phi_g\) on \(L_1\), we use 
  the fact that \(L_\infty \subseteq L_1\) is dense for all \textit{finite} measures \(\mu\). As \(\mu\) 
  is \(\sigma\)-finite, let \((A_n)\) be a measurable partition of \(\Omega\) of finite measures. Then 
  For all \(\psi \in L_1(\mu)^*\), as \(\mu_n = \mu|_{A_n}\) is finite, there exists some \(g_n \in L_\infty(\mu_n)\)
  such that 
  \[\psi_n(f) = \psi|_{A_n}(f) = \int_{A_n} fg_n \dd \mu_n = \int g_n f \dd\mu.\]
  Now, as \(\phi\) is isometric, \(\|g_n\|_\infty = \|\psi_n\| \le \|\psi\|\). Hence, taking \(g = g_n\) 
  on \(A_n\), \(g \in L_\infty\) and \(\phi_g = \psi\) as required. 
\end{proof} 

\begin{corollary}
  For all \(1 < p < \infty\), \(L_p(\mu)\) is reflexive. 
\end{corollary}
\begin{proof}
  The previous theorem provides the isometric isomorphism 
  \[\phi : L_q \to L_p^*, \langle f, \phi(g)\rangle = \int fg \dd \mu.\]
  Then, its dual (see example sheet 1) \(\phi^* : (L_p^*)^* \to L_q^*\) is also an isometric 
  isomorphism. Now, denoting \(\psi : L_p \to L_q^*\) the isometric isomorphism from \(L_p\) to 
  \(L_q^*\) (constructed the same way as \(\phi\)), It suffices to show that 
  \((\phi^*)^{-1} \circ \psi : L_p \to (L_p^*)^*\) is the canonical embedding. Indeed, 
  for all \(f \in L_p, g \in L_q\), 
  \[\langle g, \phi^*(\hat f) \rangle = \langle \phi(g), \hat f\rangle = 
  \langle f, \phi(g) \rangle = \int fg \dd \mu = \langle g, \psi(f) \rangle,\]
  so \(\phi^*(\hat f) = \psi(f)\) as claimed.
\end{proof}

\subsection{Dual of \texorpdfstring{\(C(K)\)}{C(K)}}

\subsubsection{Preliminary definitions}

For this section, we take \(K\) to be a compact Hausdorff space and introduce the following 
notations:
\begin{itemize}
  \item \(C(K) = \{f : K \to \mathbb{C} \mid f \text{ continuous}\}\) equipped with the sup-norm;
  \item \(C^\mathbb{R}(K) = \{f : K \to \mathbb{R} \mid f \text{ continuous}\};\)
  \item \(C^+(K) = \{f \in C^\mathbb{R}(K) \mid f \ge 0\};\)
  \item \(M(K) = C(K)^*;\)
  \item \(M^\mathbb{R}(K) = \{\phi \in M(K) \mid \forall \phi \in C^\mathbb{R}(K), \phi(f) \in \mathbb{R}\};\)
  \item \(M^+(K) = \{\phi : C(K) \to \mathbb{C} \mid \phi \text{ linear and } \forall f \in C^+(K), \phi(f) \ge 0\}.\)
\end{itemize}

We call elements of \(M^+(K)\) positive linear functionals.

\begin{lemma}
  Given \(\phi \in M(K)\), there exists unique \(\phi_1, \phi_2 \in M^\mathbb{R}(K)\) such that \(\phi = \phi_1 + i\phi_2\).
\end{lemma}
\begin{proof}
  \textit{Uniqueness}: We observe for \(f \in C^\mathbb{R}(K)\), if \(\phi = \phi_1 + i\phi_2\), 
  \[\phi(f) = \phi_1(f) + i \phi_2(f) \text{ and } \overline{\phi(f)} = \phi_1(f) - i\phi_2(f).\]
  Thus, 
  \[\begin{cases}
    \phi_1(f) = \text{Re}(\phi(f)) = \frac{\phi(f) + \overline{\phi(f)}}{2}, \\
    \phi_2(f) = \text{Im}(\phi(f)) = \frac{\phi(f) - \overline{\phi(f)}}{2i},
  \end{cases}\]
  so \(\phi_1\) and \(\phi_2\) are uniquely determined by \(\phi\) on \(C^\mathbb{R}(K)\) and hence 
  also on \(C(K) = C^\mathbb{R}(K) + iC^\mathbb{R}(K)\).

  \textit{Existence}: This works: 
  \[\begin{cases}
    \phi_1(f) = \frac{\phi(f) + \overline{\phi(\bar f)}}{2}, \\
    \phi_2(f) = \frac{\phi(f) - \overline{\phi(\bar f)}}{2i}.
  \end{cases}\]
\end{proof}

\begin{lemma}
  The map 
  \[\phi \mapsto \phi|_{C^\mathbb{R}(K)} : M^\mathbb{R}(K) \to C^\mathbb{R}(K)^*\]
  is an isometric isomorphism.
\end{lemma}
\begin{proof}
  Take \(\phi \in M^\mathbb{R}(K)\). It is clear that \(\|\phi|_{C^\mathbb{R}(K)}\| \le \|\phi\|\).
  On the other hand, for \(f \in C(K)\), take \(\lambda \in S^1 \subseteq \mathbb{C}\) such that 
  \(\lambda\phi(f) = |\phi(f)|\). Then, 
  \[|\phi(f)| = \phi(\lambda f) = \phi(\text{Re}(\lambda f)) + i \phi(\text{Im}(\lambda f)).\]
  However, as the left hand side is real, \(\phi(\text{Im}(\lambda f)) = 0\) and so 
  \[|\phi(f)| = \phi(\text{Re}(\lambda f)) \le \|\phi|_{C^\mathbb{R}(K)}\| \|\text{Re}(\lambda f)\|
    = \|\phi(f)\| \|f\|\]
  proving isometry.

  To prove subjectivity, take \(\psi \in C^\mathbb{R}(K)\). Then, defining 
  \[\phi(f) = \phi(\text{Re}(f)) + i\psi(\text{Im}(f))\]
  for all \(f \in C(K)\). It is clear \(\phi \in M(K)\) and \(\phi|_{C^\mathbb{R}(K)} = \psi\)
  as required.
\end{proof}

\begin{lemma}
  \(M^+(K) \subseteq M(K)\) (and in particular are continuous) and 
  \[M^+(K) = \{\phi \in M(K) \mid \|\phi\| = \phi(1_K)\}.\]
\end{lemma}
\begin{proof}
  Let \(\phi \in M^+(K)\) and \(f \in C^\mathbb{R}(K), \|f\|_\infty \le 1\) so that \(1_K \pm f \ge 0\).
  Then,
  \[0 \le \phi(1_K \pm f) = \phi(1_K) \pm \phi(f)\]
  implying \(\phi(1_K) \ge |\phi(f)|\) and hence \(\|\phi|_{C^\mathbb{R}(K)}\| = \phi(1_K)\).
  Thus, by the previous lemma, \(\phi \in M^\mathbb{R}(K)\) with \(\|\phi\| = \phi(1_K)\), i.e. 
  we've shown
  \[M^+(K) \subseteq \{\phi \in M(K) \mid \|\phi\| = \phi(1_K)\}\]

  Now, suppose \(\phi \in M(K)\) is such that \(\|\phi\| = \phi(1_K)\), we want to show 
  \(\phi \in M^+(K)\). WLOG. assume \(\|\phi\| = 1\). Then, taking \(f \in C^\mathbb{R}(K)\), 
  \(\|f\|_\infty \le 1\), let us denote \(\phi(f) = a + ib\) for some \(a, b \in \mathbb{R}\). 
  Observe, for \(t \in \mathbb{R}\), 
  \[|\phi(f + it1_K)|^2 = |a + (b + t)i|^2 = a^2 +b^2 + 2bt + t^2\]
  while on the other hand, 
  \[|\phi(f + it1_K)|^2 \le \|\phi\|^2 \|f + it1_K\|^2 \le 1 + t^2\]
  and so \( a^2 +b^2 + 2bt \le 1\) for all \(t\) which is only possible if \(b = 0\). Thus, 
  \(\phi\) takes value in \(\mathbb{R}\). 

  Now taking \(f \in C^+(K)\), \(\|f\|_\infty \le 1\), we have \(0 \le f \le 1_K\) and so 
  \[-1_K \le 1_K - 2f \le 1_K\]
  implying \(\|1_K - 2f\|_\infty \le 1\). Hence 
  \[1 - 2\phi(f) = \phi(1_K - 2f) \le 1\]
  implying \(\phi(f) \ge 0\) and so \(\phi \in M^+(K)\) as claimed.
\end{proof}

\begin{lemma}
  For all \(\phi \in M^\mathbb{R}(K)\), there exists unique \(\phi^+, \phi^- \in M^+(K)\) such that 
  \(\phi = \phi^+ - \phi^-\) and \(\|\phi\| = \|\phi^+\| + \|\phi^-\|\).
\end{lemma}
\begin{proof}
  \textit{Existence}: Define \(\phi^+\) on \(C^+(K)\) as follows: for all \(f \in C^+(K)\), take 
  \[\phi^+(f) = \sup\{\phi(g) \mid g \in C^+(K), g \le f\}.\]
  It is clear that \(\phi^+(f) \ge \phi(0) = 0\) and \(\phi^+(f) \ge \phi(f)\). Furthermore, 
  \(\phi^+\) is additive since for all \(f_1, f_2 \in C^+(K)\), \(0 \le g_1 \le f_1\) and 
  \(0 \le g_2 \le f_2\), we have 
  \[\phi^+(f_1 + f_2) \ge \phi(g_1 + g_2) = \phi(g_1) + \phi(g_2).\]
  Hence, taking the supremum over \(g_1\) and \(g_2\) provides 
  \[\phi^+(f_1 + f_2) \ge \phi^+(f_1) + \phi^+(f_2).\]
  One the other hance, given \(0 \le g \le f_1 + f_2\),  
  \[\phi(g) = \phi(g \wedge f_1) + \phi(g - (g \wedge f_1)) \le \phi^+(f_1) + \phi^+(f_2)\]
  since \(g \wedge f_1 \le f_1\) and \(g - (g \wedge f_1) \le g - f_1 \le f_2\). 

  Now, we define \(\phi^+\) on \(C^\mathbb{R}(K)\). such that for all \(f \in C^\mathbb{R}(K)\), by
  writing \(f = f^+ - f^-, f^\pm \in C^+(K)\), we take 
  \[\phi^+(f) = \phi^+(f^+) - \phi^+(f^-).\]
  Finally, to define \(\phi^+\) on \(C(K)\), for all \(f \in C(K)\), we take 
  \[\phi^+(f) = \phi^+(f_1) + i\phi^+(f_2)\]
  where \(f_1, f_2 \in C^\mathbb{R}(K)\) are such that \(f = f_1 + if_2\).

  Of course, now we've defined \(\phi^+ \in M^+(K)\), we take \(\phi^- = \phi^+ - \phi\). 
  \(\phi^- \in M^+(K)\) also, since for all \(f \in C^+(K)\), 
  \[\phi^-(f) = \phi^+(f) - \phi(f) \ge 0.\]
  It remains to show \(\|\phi\| = \|\phi^+\| + \|\phi^-\|\). Indeed, by considering
  \begin{equation}\label{eq:phi+phi-}
    \|\phi\| \le \|\phi\|^+ + \|\phi^-\| = \phi^+(1_K) + \phi^-(1_K) = 2\phi^+(1_K) - \phi(1_K).
  \end{equation}
  On the other hand, for all \(0 \le f \le 1_K\), we have 
  \[-1_K \le 2f - 1_K \le 1_f\]
  and so \(\|2f - 1_K\|_\infty \le 1\). Hence, 
  \[2\phi(f) - \phi(1_K) = \phi(2f - 1_K) \le \|\phi\|.\]
  Thus, taking the supremum over \(f\), the right hand side of equaton~\eqref{eq:phi+phi-} is less 
  equal to the operator norm of \(\phi\) implying 
  \[\|\phi\| = \|\phi\|^+ + \|\phi^-\|\]
  as required.

  \textit{Uniqueness}: Suppose \(\phi = \psi_1 - \psi_2\) and \(\|\phi\| = \|\psi_1\| + \|\psi_2\|\) 
  for some \(\psi_1, \psi_2 \in M^+(K)\). Then, for all \(0 \le g \le f\), 
  \[\phi(g) = \psi_1(g) - \psi_2(g) \le \psi_1(g) \le \psi_1(f),\]
  implying \(\psi_1 \ge \phi^+\) and \(\psi_1 - \phi^+ \in M^+(K)\). Thus, we also have
  \(\psi_2 - \phi^- = \psi_1 - \phi^+ \in M^+(K)\). Then, 
  \begin{align*}
    \|\psi_1 - \phi^+\| + \|\psi_2 - \phi^-\| &= \psi_1(1_K) - \phi^+(1_K) + \psi_2(1_K) - \phi^-(1_K)\\
    & = (\psi_1(1_K) + \psi_2(1_K)) - (\phi^+(1_K) + \phi^-(1_K))\\
    & = (\|\psi_1\| + \|\psi_2\|) - (\|\phi^+\| + \|\phi^-\|)\\
    & = \|\phi\| - \|\phi\| = 0
  \end{align*} 
  providing uniqueness.
\end{proof}

\subsubsection{Topological preliminaries}

We recall the following facts:
\begin{itemize}
  \item \(K\) is said to be normal if for all disjoint closed subsets \(E_1, E_2 \subseteq K\), 
    there exists disjoint open sets \(U_1, U_2 \subseteq K\) such that \(E_1 \subseteq U_1\) and
    \(E_2 \subseteq U_2\).

    Equivalently, if \(E \subseteq U \subseteq K\) are such that \(E\) is closed and \(U\) is open, 
    then there exists a open \(V\) such that \(E \subseteq V \subseteq \overline{V} \subseteq U\).
  \item \textit{Urysohn's lemma}: Given disjoint closed subsets \(E_1, E_2 \subseteq K\), there exists 
    a continuous function \(f : K \to [0, 1]\) such that \(f = 0\) on \(E_1\) and \(f = 1\) on \(E_2\).

\end{itemize}
  
\textit{Notations}: \(f \prec U\) denotes the fact that 
\begin{itemize}
  \item \(U\) is open,
  \item \(f : K \to [0, 1]\) is continuous,
  \item and \(\text{supp}(f) = \overline{\{x \mid f(x) \neq 0\}} \subseteq U\).
\end{itemize}
On the other hand, \(E \prec f\) denotes
\begin{itemize}
  \item \(E\) is closed;
  \item \(f : K \to [0, 1]\) is continuous,
  \item and \(f = 1\) on \(E\).
\end{itemize}

Using this notation, Urysohn's lemma provides the existence of a \(f\) such that \(E \prec f \prec U\).

\begin{lemma}
  Let \(E \subseteq K\) be closed and let \(U_j \subseteq K\) be open sets such 
  that \(E \subseteq \bigcup_{j = 1}^n U_k\). Then, 
  \begin{itemize}
    \item there exists open \(V_j\) such that \(\overline{V_j} \subseteq U_j\) and 
      \(E \subseteq \bigcup_{j = 1}^n V_j\).
    \item there exists \(f_j \prec U_j\) such that \(\sum_{j = 1}^n f_j \le 1\) on \(K\) and 
    \(\sum_{j = 1}^n f_j = 1\) on \(E\).
  \end{itemize}
\end{lemma}
\begin{proof}
  For the first part we induct on \(n\). 

  Since \(E \setminus U_n \subseteq \bigcup_{j = 1}^{n - 1} U_j\) and \(E \setminus U_n\) is closed, 
  we can apply the inductive hypothesis to obtain \(V_j\) for \(j = 1, \cdots, n - 1\) such that 
  \(\overline{V_j} \subseteq U_j\) and \(E \setminus U_n \subseteq \bigcup_{j = 1}^{n - 1} V_j\).
  Then,
  \[E \setminus \bigcup_{j = 1}^{n - 1} V_j \subseteq U_n,\] 
  and thus, by the normality of \(K\), there exists some open set \(V\), such that 
  \[E \setminus \bigcup_{j = 1}^{n - 1} V_j \subseteq V \subseteq \overline{V} \subseteq U_n.\]
  Hence, we have \(E \subseteq \bigcup_{j = 1}^n V_j\) where \(\overline{V_j} \subseteq U_j\) for all \(j\).

  For the second part, choose \(V_j\) as in the first part. Then, by Urysohn's lemma, there exists 
  \(g_j\) such that \(\overline{V_j} \prec g_j \prec U_j\) and 
  \(K \setminus \bigcup V_j \prec g_0 \prec K \setminus E\). So, defining 
  \(g = \sum_{j = 0}^n g_j\), \(g\) is continuous and satisfies \(g \ge 1\) on \(K\). Thus, 
  setting \(f_j = g_j / g\) for \(j = 1, \cdots, n\), we have \(f_j : K \to [0, 1]\) is continuous and satisfies 
  \[\sum_{j = 1}^n f_j = \sum_{j = 1}^n \frac{g_j}{g} \le 1\]
  on \(K\), and by noting \(g_0 = 0\) on \(E\),
  \[\sum_{j = 1}^n f_j = \sum_{j = 1}^n \frac{g_j}{g} = 1\]
  on \(E\).
\end{proof}

\begin{definition}[Regular]
  A Borel measure \(\mu\) on the Borel space \(X\) is said to be regular if 
  \begin{itemize}
    \item for all compact \(E \subseteq X\), \(\mu(E) < \infty\),
    \item for all \(A \in \mathcal{B}(X)\), 
      \[\mu(A) = \inf\{\mu(U) \mid A \subseteq U \in \mathcal{G}\}\]
      where \(\mathcal{G}\) is the collection of all open sets in \(X\).
    \item for all \(U \in \mathcal{G}\),
      \[\mu(U) = \sup\{\mu(E) \mid E \subseteq U, E \text{ compact}\}.\]
  \end{itemize}
  A compact measure \(\nu\) is said to be regular if \(|\nu|\) is.
\end{definition}

\begin{proposition}
  If \(X\) is compact Hausdorff, then TFAE:
  \begin{itemize}
    \item The Borel measure \(\mu\) is regular;
    \item \(\mu(X) < \infty\) and for all \(A \in \mathcal{B}(X)\), 
      \[\mu(A) = \inf\{\mu(U) \mid A \subseteq U \in \mathcal{G}\};\]
    \item \(\mu(X) < \infty\) and for all \(A \in \mathcal{B}(X)\), 
      \[\mu(A) = \sup\{\mu(E) \mid E \subseteq A, E \text{ closed}\}.\]
  \end{itemize}
\end{proposition}

\subsubsection{Riesz-Markov representation theorem}

If \(\nu\) is a complex Borel measure on \(K\), for any \(f \in C(K)\), we have \(f\) is Borel-measurable 
and by observing
\[\int |f| \dd |\nu| \le \|f\|_\infty |\nu|(K) < \infty,\]
\(f\) is also \(\nu\)-integrable. Thus, we may define the bounded linear functional
\[\phi : C(K) \to \mathbb{C} : f \mapsto \int f \dd \nu.\]
\(\phi\) is clearly linear and it is bounded since 
\[|\phi(f)| \le \int |f| \dd |\nu| \le \|f\|_\infty |\nu|(K)\]
so \(\phi \in M(K) = C(K)^*\) and \(\|\phi\| \le \|\nu\|_1\). If \(\nu\) is a signed measure, then 
\(\phi \in M^\mathbb{R}(K) \simeq C^\mathbb{R}(K)^*\) and if \(\nu\) is a positive measure, then 
\(\phi \in M^+(K)\). It turns out that the converse is also true, namely elements of \(M(K)\) can also 
be represented by complex measures.

\begin{theorem}[Riesz-Markov representation]
  Given \(\phi \in M^+(K)\), there exists a unique regular Borel measure \(\mu\) on \(K\) such that 
  for all \(f\) \(\mu\)-integrable, \(\phi(f) = \int f \dd\mu\). Furthermore, \(\|\phi\| = \mu(K) = \|\mu\|_1\).
\end{theorem}
\begin{proof}
  \textit{Uniqueness}: Suppose we have two regular Borel measures \(\mu_1, \mu_2\) both representing \(\phi\) 
  in the sense as above. Then for all \(E \subseteq U \subseteq K\) with \(E\) closed and \(U\) open, 
  by Urysohn's lemma, there exists some \(f : K \to [0, 1]\) such that \(E \prec f \prec U\). Hence, 
  \[\mu_1(E) \le \int f \dd \mu_1 = \phi(f) = \int f \dd\mu_2 \le \mu_2(U).\]
  So, as both \(\mu_1\) and \(\mu_2\) are regular, this implies \(\mu_1 \le \mu_2\). By symmetry, we 
  also have \(\mu_2 \le \mu_1\) providing the uniqueness.

  \textit{Existence}: We would like to define a measure akin to \(\mu(A) = \phi(1_A)\). However, 
  as \(1_A\) is not continuous, we will approximate this construction by defining an outer measure \(\mu^*\).
  
  Given \(U \in \mathcal{G}\) (recall that \(\mathcal{G}\) is the set of all open sets in \(K\)), we 
  define 
  \[\mu^*(U) := \sup \{\phi(f) \mid f \prec U\}.\]
  Observe straightaway that \(\mu^*(\varnothing) = 0\) and \(\mu^*(K) = \phi(1_K) = \|\phi\|\). 
  
  We will now show \(\mu^*\) satisfies sub-\(\sigma\)-additivity. Suppose we have \(U \subseteq \bigcup_{k = 1}^\infty U_k\) 
  for some \(U, U_k \in \mathcal{G}\). Then, given \(f \prec U\), by compactness, there exists some \(n\) 
  such that 
  \[\text{supp}(f) \subseteq \bigcup_{k = 1}^n U_k.\]
  By the partition of unity, for each \(k = 1, \cdots, n\), there exists some \(h_k \prec U_k\) such that 
  \(\sum h_k \le 1\) on \(K\) and \(\sum h_k = 1\) on \(\text{supp}(f)\). Thus, 
  \[\phi(f) = \phi\left(\sum_{k = 1}^n h_k f\right) = \sum_{k = 1}^n \phi(h_k f) 
    \le \sum_{k = 1}^n \mu^*(U_k) \le \sum_{k = 1}^\infty \mu^*(U_k).\]
  Hence, as this inequality holds for all \(f \prec U\), we have \(\mu^*(U) \le \sum_{k = 1}^\infty \mu^*(U_k)\).
  Furthermore, it follows that given \(U, V \in \mathcal{G}\), \(U \subseteq V\), we have \(\mu^*(U) \le \mu^*(V)\) 
  and so, 
  \[\mu^*(U) = \inf \{\mu^*(V) \mid U \subseteq V \in \mathcal{G}\}.\]
  With this in mind, we extend \(\mu^*\) to all of \(2^K\) by defining 
  \[\mu^*(A) = \inf \{\mu^*(V) \mid A \subseteq V \in \mathcal{G}\}\]
  for any \(A \subseteq K\).

  Again, it is clear that \(\mu^*(\varnothing) = 0\) and \(\mu^*(K) = \|\phi\|\). For sub-\(\sigma\)-additivity,
  let \(A \subseteq \bigcup_{k = 1}^\infty A_n\). Then, for any \(\epsilon > 0\), for each \(n\), we may choose 
  \(U_n \in \mathcal{G}\) such that \(A_n \subseteq U_n\) and 
  \[\mu^*(U_n) < \mu^*(A_n) + \epsilon 2^{-n}.\]
  Hence, \(A \subseteq \bigcup_{k = 1}^\infty U_k\) and so,
  \[\mu^*(A) \le \mu^*\left(\bigcup_{k = 1}^\infty U_k\right) \le \sum_{k = 1}^\infty \mu^*(U_k) 
    \le \sum_{k = 1}^\infty \mu^*(A_k) + \epsilon.\]
  Thus, as \(\epsilon > 0\) was arbitrary, it follows \(\mu^*(A) \le \sum_{k = 1}^\infty \mu^*(A_k)\) 
  and \(\mu^*\) is an outer measure on \(K\). 
  
  Now, by Carathéodory extension, \(\mu^*\) restricts to a 
  measure on the set of sets which are \(\mu^*\)-measurable. Thus, by showing all open sets of \(K\) 
  are \(\mu^*\)-measurable, we may restrict \(\mu^*\) on to \(\mathcal{B}(K)\) to obtain the desired 
  Borel measure. Take \(U \in \mathcal{G}\) and \(A \subseteq K\), we need to show 
  \[\mu^*(A) \ge \mu^*(A \cap U) + \mu^*(A \setminus U).\]
  First, let us consider the cas that \(A = V \in \mathcal{G}\). Then, taking \(f \prec U \cap V\) 
  and \(g \prec V \setminus \text{supp}(f)\) so that \(f, g\) are disjointedly supported on \(V\), 
  we have \(f + g \prec V\) and so, 
  \[\mu^*(V) \ge \phi(f + g) = \phi(f) + \phi(g).\]
  Taking the supremum over \(g\), we have 
  \[\mu^*(V) \ge \phi(f) + \mu^*(V \setminus \text{supp}(f)) \ge \phi(f) + \mu^*(V \setminus U).\]
  Now, taking the supremum over \(f\), 
  \[\mu^*(V) \ge \mu^*(U \cap V) + \mu^*(V \setminus U)\]
  as required. 

  For general \(A\), let \(V \in \mathcal{G}\) such that \(A \subseteq V\). Then, 
  \[\mu^*(V) \ge \mu^*(V \cap U) + \mu^*(V \setminus U) \ge \mu^*(A \cap U) + \mu^*(A \setminus U).\]
  Hence, taking the infimum over \(V\), it follows 
  \[\mu^*(A) \ge \mu^*(A \cap U) + \mu^*(A \setminus U)\]
  as required.

  Thus, \(\mu := \mu^*|_{\mathcal{B}}\) is a Borel measure on \(K\) and it is regular by construction. 
  It remains to show that \(\mu\) represents \(\phi\). It is sufficient to show \(\phi(f) \le \int f \dd \mu\) 
  for all \(f \in C^\mathbb{R}(K)\) since if this holds, then 
  \[- \phi(f) = \phi(-f) \le \int -f \dd \mu = - \int f \dd \mu\]
  providing the reverse inequality. 

  Let \(f \in C^\mathbb{R}(K)\) and choose \(a < b \in \mathbb{R}\) so that \(f(K) \subseteq [a, b]\). 
  WLOG. assume \(a > 0\) and fix \(\epsilon > 0\) and choose 
  \[0 < y_0 < a < y_1 < \cdots < y_n = b\]
  such that \(y_j - y_{j - 1} < \epsilon\). Let \(A_j := f^{-1}((y_{j - 1}, y_j])\) so 
  \(K = \bigcup_{j = 1}^n\) is a Borel partition of \(K\). For each \(j\), choose \(U_j \in \mathcal{G}\) 
  such that \(A_j \subseteq U_j \subseteq f^{-1}((y_{j - 1}, y_j + \epsilon))\) and 
  \[\mu(U_j) < \mu(A_j) + \frac{\epsilon}{n}.\]
  Then by the partition of unity, there exists \(h_j \prec U_j\) such that \(\sum_{j = 1}^n h_j = 1_K\) 
  so
  \begin{align*}
    \phi(f) & = \sum_{j = 1}^n \phi(h_j f) \le \sum \phi((y_j + \epsilon) h_j) = \sum (y_j + \epsilon)\phi(h_j)\\
      & \le \sum (y_j + \epsilon)\mu(U_j) \le \sum (y_j + \epsilon)\left(\mu(A_j) + \frac{\epsilon}{n}\right)\\
      & = \int \sum y_j 1_{A_j} \dd\mu + 2 \epsilon \mu(K) + (b + 2\epsilon)\epsilon\\
      & \le \int f \dd \mu + C \epsilon.
  \end{align*} 
  Hence, as \(\epsilon\) was arbitrary, \(\phi(f) \le \int f \dd \mu\) are required.
\end{proof}

\begin{corollary}
  For all \(\phi \in M(K)\), there exists a unique regular Borel complex measure \(\nu\) such that 
  for all \(f \in C(K)\), \(\phi(f) = \int f \dd \nu\) and \(\|\phi\| = \|\nu\|_1\). Furthermore, if 
  \(\phi \in M^\mathbb{R}(K)\), then \(\nu\) is a signed measure.
\end{corollary}
\begin{proof}
  Existence follows by Jordan decomposition while uniqueness follows from \(\|\phi\| = \|\nu\|_1\). 
  We will show \(\|\phi\| = \|\nu\|_1\). We've seen that \(\|\phi\| \le \|\nu\|_1\) so it remains to 
  show the reverse. Recall that 
  \[\|\nu\| = |\nu|(K) = \sup \left\{\sum_{j = 1}^n |\nu(A_j)| \mid (A_j)_{j = 1}^n \text{ is a Borel partition of } K\right\}.\]
  So, taking \((A_j)\) a Borel partition of \(K\), for each \(j\) let us choose \(E_j\) closed such that 
  \(E_j \subseteq A_j\) and 
  \[|\nu|(A_j \setminus E_j) < \frac{\epsilon}{n}\]
  which exists by regularity. Noting that \(E_j \subseteq K \setminus \bigcup_{i \neq j} E_i\) which is 
  open, there exists some open \(U_j\) such that \(E_j \subseteq U_j \subseteq K \setminus \bigcup_{i \neq j} E_i\)
  and 
  \[|\nu|(U_j \setminus E_j) < \frac{\epsilon}{n}.\]
  Then, \(E := \bigcup_{j = 1}^n E_j \subseteq \bigcup_{j = 1}^n U_j\) and by the partition of unity, 
  there exists \(h_j \prec U_j\) such that \(\sum h_j \le 1\) on K and \(\sum h_j = 1\) on \(E\). 
  Now, as \(E_j\) are disjoint, \(h_j = 1\) on \(E_j\). Thus, choosing \(\lambda_j \in \mathbb{C}\), 
  \(|\lambda| = 1\) such that \(|\nu|(E_j) = \lambda_j \nu(E_j)\), we have 
  \begin{align*}
    \left|\sum |\nu(E_j)| - \phi\left(\sum\lambda h_j\right)\right| 
      & = \left|\sum \lambda_j \int (1_{E_j} - h_j) \dd \nu\right|\\
      & \le \sum \int |1_{E_j} - h_j| \dd \nu \le \sum |\nu|(U_j \setminus E_j) < \epsilon.
  \end{align*}
  Hence, 
  \begin{align*}
    \sum |\nu(A_j)| & \le \sum |\nu(E_j)| + \epsilon \\
    & \le \left|\phi\left(\sum \lambda_j 1_{E_j}\right)\right| 
      \le \|\phi\| \left\| \sum \lambda_j h_j\right\|_\infty + 2\epsilon 
      \le \|\phi\| + 2\epsilon,
  \end{align*}
  implying \(\|\nu\|_1 = \|\phi\|\) as required.
\end{proof}

\begin{corollary}
  The space of regular complex Borel measures is a complex Banach space with the total variation 
  norm and it is isometrically isomorphic to \(M(K) = C(K)^*\).
\end{corollary}

\newpage
\section{Weak Topology}

\subsection{General weak topology}

Let \(X\) be a set and \(\mathcal{F}\) be a collection of functions such that for each \(f \in \mathcal{F}\), 
\(f : X \to Y_f\) where \(Y_f\) is a topological space. Then the weak topology \(\sigma(X, \mathcal{F})\) 
is the smallest topological space such that for all \(f \in \mathcal{F}\), \(f\) is continuous.
We have the following straight forward properties about the weak topology.

\begin{proposition} Taking \(X, \mathcal{F}\) as above,
  \begin{itemize}
    \item \(S := \{f^{-1}(U) \mid f \in \mathcal{F}, U \text{ open in } Y_f\}\) generates \(\sigma(X, \mathcal{F})\).
    \item \(V \subseteq X\) is open in \(\sigma(X, \mathcal{F})\) iff for all \(x \in V\), there exists 
    \(f_1, \cdots, f_n \in \mathcal{F}\) and open sets \(U_1, \cdots, U_n\) such that \(U_i \subseteq Y_{f_i}\)
    and 
    \[x \in \bigcap_{j = 1}^n f^{-1}_j(U_j) \subseteq V.\]
    \item If \(S_f\) generates the topology of \(Y_f\) for all \(f \in \mathcal{F}\), then 
    \(\{f^{-1}(U) \mid U \in S_f, f \in \mathcal{F}\}\) generates \(\sigma(X, \mathcal{F})\).
    \item If \(Y_f\) is Hausdorff for all \(f \in \mathcal{F}\) and \(\mathcal{F}\) separates points, 
    then, so is \(\sigma(X, \mathcal{F})\) Hausdorff.
    \item If \(Y \subseteq X\), then \(\sigma(X, \mathcal{F})|_Y = \sigma(Y, \mathcal{F}|_Y)\).
    \item (universal property) Given \(Z\) a topological space and \(g : Z \to X\), then \(g\) is continuous 
    with respect to \(\sigma(X, \mathcal{F})\) iff for all \(f \in \mathcal{F}\), \(f \circ g : Z \to Y_f\)
    is continuous.
  \end{itemize}
\end{proposition}

The weak topology generalizes the subspace topology by considering \(\sigma(Y, \{\iota\})\) for 
\(\iota : Y \hookrightarrow X\) the inclusion map and the product topology which has the topology 
\[\sigma\left(\prod_{\gamma \in \Gamma} X_{\gamma}, \{\pi_{\gamma} \mid \gamma \in \Gamma\}\right),\]
where \(\pi_{\gamma} : \prod_{\gamma \in \Gamma} X_{\gamma} \to X_{\gamma}\) is the projection map.

\begin{proposition}
  Let \(X\) be a set and for each \(n \in \mathbb{N}\), let \((Y_n, d_n)\) be metric spaces. Then, if 
  \(\mathcal{F} := \{f_n : X \to Y_n \mid n \in \mathbb{N}\}\) separates points, then \(\sigma(X, \mathcal{F})\)
  is metrizable.
\end{proposition}
\begin{proof}
  WLOG. by replacing \(d_n\) by \(d_n \wedge 1\) which is equivalent, we may assume that \(d_n \le 1\). 
  Then, it is easy to check that 
  \[d(x, y) := \sum_{n = 1}^\infty 2^{-n} d_n(f_n(x), f_n(y)),\]
  form a metric on \(X\). 

  By noting that any \(f_n\) in the above proposition is Lipschitz with respect to the topology generated 
  by \(d\), \(f_n\) is \(\mathcal{T}_d\)-continuous and so \(\sigma(X, \mathcal{F}) \subseteq \mathcal{T}_d\). Conversely, 
  as each \(f_n\) is \(\sigma(X, \mathcal{F})\)-continuous, the map 
  \[(x, y) \mapsto d_n(f_n(x), f_n(y))\]
  is also \(\sigma(X, \mathcal{F})\)-continuous. Hence, by the Weierstrass-M-test, it follows \(d\) is also 
  \(\sigma(X, \mathcal{F})\)-continuous implying 
  \[\mathcal{T}_d = \sigma(X, \mathcal{F})\]
  as required.
\end{proof}

\begin{theorem}[Tychonov]
  The product of compact spaces is compact in the product topology.
\end{theorem}
\begin{proof}
  Let \(\Gamma\) be the index set and for each \(\gamma \in \Gamma\), let \(X_\gamma\) be a 
  compact space and denote \(X = \prod_{\gamma \in \Gamma} X_\gamma\). We will show \(X\) is 
  compact by showing: for any non-empty family of closed sets \(\mathcal{A}\) with the 
  finite intersection property (fip.), that is, for all \(A_1, \cdots, A_n \in \mathcal{A}\),
  we have \(\bigcap_{i = 1}^n A_i \neq \varnothing\), then \(\bigcap_{A \in \mathcal{A}} A \neq \varnothing\). 

  By Zorn's lemma, there exists a maximal family \(\mathcal{B}\) of (not necessarily closed) subsets of 
  \(X\) with fip. and satisfies \(\mathcal{A} \subseteq \mathcal{B}\). Then, 
  \[\bigcap_{A \in \mathcal{A}} A \supseteq \bigcap_{B \in \mathcal{B}} B\]
  and so, it suffices to show \(\bigcap_{B \in \mathcal{B}} B \neq \varnothing\).

  By the maximality of \(\mathcal{B}\), we observe that if \(A \subseteq X\) satisfies 
  \(A \cap B \neq \varnothing\) for all \(B \in \mathcal{B}\), then \(A \in B\). 
  Fix \(\gamma \in \Gamma\), \(\{\pi_\gamma(B) \mid B \in \mathcal{B}\}\) has fip. and 
  hence, as \(X_\gamma\) is compact, it follows 
  \(\bigcap_{B \in \mathcal{B}} \overline{\pi_{\gamma}^{-1}(B)} \neq \varnothing\).
  Choose \(x_\gamma \in \bigcap_{B \in \mathcal{B}} \overline{\pi_{\gamma}^{-1}(B)}\), 
  we will show \(x = (x_\gamma)_{\gamma \in \Gamma} \in \bigcap_{B \in \mathcal{B}} \overline{B}\).
  Let \(V\) be an open neighborhood of \(x\), we need to show \(V \cap B \neq \varnothing\) for all 
  \(B \in \mathcal{B}\). WLOG. write 
  \[V = \bigcap_{i = 1}^n \pi^{-1}_{\gamma_i}(U_i)\]
  for some \(\gamma_1, \cdots, \gamma_n \in \Gamma\) and \(U_1, \cdots, U_n\) open neighborhoods of
  \(x_{\gamma_i}\).

  Since \(x_{\gamma_i} \in \bigcap_{B \in \mathcal{B}} \pi_{\gamma_i}^{-1}(B)\), we have
  \(U_{\gamma_i} \cap \pi_{\gamma_i}(B) \neq \varnothing\) for all \(B \in \mathcal{B}\). 
  Thus, by maximality, we have \(\pi_{\gamma_i}^{-1}(U_i) \in \mathcal{B}\) and so 
  \[V = \bigcap_{i = 1}^n \pi_{\gamma_i}^{-1}(U_i) \in \mathcal{B}\]
  implying \(V \cap B \neq \varnothing\) for all \(B \in \mathcal{B}\). Hence, as \(V\) was chosen 
  arbitrarily, \(x \in B\) for all \(B \in \mathcal{B}\) as required.
\end{proof}

\subsection{Weak topology on vector spaces}

Let \(E\) be a real or complex vector space and \(F\) a subspace of the space of all linear functionals 
on \(E\) that separates points of \(E\). We will in this section consider \(\sigma(E, F)\). 
We recall that \(U \subseteq E\) is weakly open iff for all \(x \in U\), there exists 
\(f_1, \cdots, f_n \in F\), \(\epsilon > 0\) such that 
\[\{y \in E \mid |f_i(y - x)| < \epsilon, i = 1, \cdots, n\} \subseteq U.\]
For \(f \in F\), define \(p_f : E \to \mathbb{R}\) by \(p_f(x) = |f(x)|\). Then, 
\[\mathcal{P} := \{p_f \mid f \in F\}\]
is a family of semi-norms which separates points of \(E\). Thus, the weak topology on \(E\) generated 
by \(F\) is the same as the LCS topology generated by \(\mathcal{P}\).

\begin{lemma}
  Let \(E\) be a real or complex vector space and \(f, g_1, \cdots, g_n\) linear functionals on \(E\) 
  such that 
  \[\bigcap_{i = 1}^n \ker g_i \subseteq \ker f_i.\]
  Then, \(f \in \langle g_1, \cdots, g_n \rangle\).
\end{lemma}
\begin{proof}
  Define \(T : E \to \mathbb{F}^n\) by \(Tx = (g_i(x))_{i = 1}^n\). Then, \(\ker T \subseteq \ker f\).
  Thus, there exists some linear \(h : \text{Im}(T) \to \mathbb{F}\) such that \(f = h \circ T\). Thus, 
  by Hahn-Banach, extending \(h\) to \(\mathbb{F}^n \to \mathbb{F}\), we can write 
  \(h(y) = \sum_{i = 1}^n a_i y_i\) for all \(y = (y_i)_{i = 1}^n \in \mathbb{R}^n\). Hence, 
  for all \(x \in E\), 
  \[f(x) = h(Tx) = \sum_{i = 1}^n a_i g_i(x),\]
  implying \(f \in \langle g_1, \cdots, g_n \rangle\) as required.
\end{proof}

\begin{proposition}
  Let \(E, F\) as above. A linear functional \(f\) on \(E\) is weakly continuous iff \(f \in F\).
  Namely, \((E, \sigma(E, F))^* = F\).
\end{proposition}
\begin{proof}
  The converse is true by definition. For the other direction, let \(f\) be a weakly continuous linear 
  functional. Then, \(V := f^{-1}(B_1(0))\) is an open neighborhood of \(0\) in \((E, \sigma(E, F))\).
  Thus, there exists \(g_1, \cdots, g_n \in F\) and \(\epsilon > 0\) such that 
  \[U := \{x \in E \mid |g_i(x)| < \epsilon, i = 1, \cdots, n\} \subseteq V.\]
  Then, for all \(x \in \bigcap_{i = 1}^n \ker g_i\), for all \(\lambda \in \mathbb{F}\) 
  such that \(\lambda x \in U \subseteq V\) and so, \(|f(\lambda x)| = |\lambda| |f(x)| < 1\) 
  implying \(x \in \ker f\). Thus, by the previous lemma \(f \in \langle g_1, \cdots, g_n \rangle\) 
  and so \(f \in F\).
\end{proof}

If \(X\) is a normed space, the weak topology on \(X\) is \(w := \sigma(X, X^*)\). By Hahn-Banach, \(X^*\) 
separates points of \(X\) and so the weak topology is Hausdorff. As before, a subset \(U \subseteq X\) 
is weakly open iff for all \(x \in U\), there exists \(f_1, \cdots, f_n \in X^*\) and \(\epsilon > 0\) 
such that
\[\{y \in X \mid |f_i(y - x)| < \epsilon, i = 1, \cdots, n\} \subseteq U.\]
Now, by identifying \(X\) in \((X^*)^*\) by the canonical embedding, we define the weak-\(*\) topology 
on \(X^*\) by \(w^* := \sigma(X^*, X)\). A subset \(U \subseteq X^*\) is weak-\(*\) open iff for all
\(f \in U\), there exists \(x_1, \cdots, x_n \in X\) and \(\epsilon > 0\) such that
\[\{g \in X^* \mid |g(x_i) - f(x_i)| < \epsilon, i = 1, \cdots, n\} \subseteq U.\]

\begin{proposition}
  \((X, w), (X^*, w^*)\) are locally convex spaces. In particular, they are Hausdorff and their 
  vector space operations are continuous. Furthermore, 
  \begin{itemize}
    \item \(w \subseteq \|\cdot\|-\text{topology}\) with equality iff \(X\) is finite dimensional.
    \item \(w^* \subseteq \sigma(X^*, (X^*)^*) \subseteq \|\cdot\|-\text{topology}\) with the first 
      inclusion becoming an equality iff \(X\) is reflexive and the second inclusion becoming an
      equality iff \(X\) is finite dimensional.
    \item if \(Y \le X\), then 
      \[\sigma(X, X^*)|_Y = \sigma(Y, \{f|_Y \mid f \in X^*\}) = \sigma(Y, Y^*)\]
      where the last equality follows by Hahn-Banach.
    \item The canonical embedding \(X \to (X^*)^*\) is a w-to-w\(^*\) homeomorphism between \(X\) 
      and \(\hat X\).
  \end{itemize}
\end{proposition}

\begin{proposition}
  Let \(X\) be a normed space. Then,
  \begin{itemize}
    \item a linear functional \(f\) on \(X\) is weakly continuous iff \(f \in X^*\).
    \item a linear functional \(\phi\) on \(X^*\) is weak-\(*\) continuous iff \(\phi \in \hat X\).
    \item \(\sigma(X^*, X) = \sigma(X^*, (X^*)^*)\) iff \(X\) is reflexive.
  \end{itemize}
\end{proposition}
\begin{proof}
  The only slightly non-trivial part is the forward direction of the third statement. But this is 
  also straight forward. Let \(\phi \in (X^*)^*\), we need to show \(\phi \in \hat X\). 
  Since \(\sigma(X^*, X) = \sigma(X^*, (X^*)^*)\), \(f\) is weak-\(*\) continuous and the result follows 
  by the second claim.
\end{proof}

\begin{definition}
  Let \(X\) be a normed space, \(A \subseteq X\) is said to be weakly bounded if 
  \(\{f(x) \mid x \in A\}\) is bounded for all \(f \in X^*\).
\end{definition}

Clearly, as all \(f \in X^*\) are bounded, bounded in \(\|\cdot\|\) implies weakly bounded.

We recall the principle of uniformly boundedness (PUB).

\begin{theorem}
  Let \(X\) be a Banach space, Y a normed space and \(\mathcal{T} \subseteq \mathcal{B}(X, Y)\). 
  Then, if \(\mathcal{T}\) is point-wise bounded, i.e. 
  \[\sup_{T \in \mathcal{T}} \|Tx\| < \infty,\]
  then \(\sup_{T \in \mathcal{T}} \|T\| < \infty\).
\end{theorem}

\begin{proposition}
  If \(X\) is a normed space, 
  \begin{itemize}
    \item if \(A \subseteq X\) is weakly bounded, then \(A\) is \(\|\cdot\|\)-bounded.
    \item if \(X\) is in addition complete, then if \(B \subseteq X^*\) is w\(^*\)-bounded, then 
      \(B\) is \(\|\cdot\|\)-bounded.
  \end{itemize}
\end{proposition}
\begin{proof}
  Firstly, defining \(\hat A = \{\hat x \mid x \in A\}\), as \(A\) is weakly bounded, for all 
  \(f \in X^*\), 
  \[\sup_{\hat x \in \hat X}\|\hat x(f)\| = \sup_{x \in A} \|f(x)\| < \infty.\]
  Thus, by PUB (note that we are using the fact \((X^*)^*\) is complete), 
  \(\sup_{x \in A} \|x\| = \sup_{\hat x \in \hat A} \|\hat x\| < \infty\) 
  as required. 
  
  On the other hand, if \(X\) i complete and \(B \subseteq X^*\) is w\(^*\)-bounded, then 
  we may directly apply PUB to obtain the bound in \(\|\cdot\|\) as required.
\end{proof}

\newpage
\section{Banach Algebra}

\subsection{Definitions}

\begin{definition}[Algebra]
  A real or complex algebra is a real or resp. complex vector space \(A\) with a multiplication
  \[A \times A \to A : (a, b) \mapsto ab,\]
  such that 
  \begin{itemize}
    \item \((ab)c = a(bc)\),
    \item \(a(b+c) = ab + ac\),
    \item \((a + b)c = ac + bc\),
    \item for all \(\lambda \in \mathbb{R}\) or resp. \(\mathbb{C}\), \(\lambda(ab) = (\lambda a)b = a(\lambda b)\).
  \end{itemize}
\end{definition}

\begin{definition}[Unital]
  An algebra is said to be \textit{unital} if there exists a \(1 \ne 0 \in A\) such that for all \(a \in A\),
  \[a1 = 1a = a.\]
  Such an element is unique and is called the unit of \(A\).
\end{definition}

\begin{definition}[Algebra norm]
  An algebra norm on an algebra \(A\) is a vector norm \(\|\cdot\|\) such that for all \(a, b \in A\), 
  \[\|ab\| \le \|a\| \|b\|.\]
\end{definition}
This property implies that multiplication is continuous wrt. the topology induced by the norm.

\begin{definition}[Normed algebra]
  A normed algebra is an algebra with an algebra norm.
\end{definition}

\begin{definition}[Banach algebra]
  A Banach algebra is a complete normed algebra.
\end{definition}

\begin{definition}[Unital normed algebra]
  A unital normed algebra is a unital algebra with a algebra norm such that \(\|1\| = 1\).
\end{definition}

If \(A\) is a unital algebra with an algebra norm \(\|\cdot\|\), then defining another norm
\[\|a\|' := \sup \{\|ab\| \mid \|b\| \le 1\}.\]
\(\|\cdot\|\) and \(\|\cdot\|'\) are equivalent and \(\|1\|' = 1\). Thus, we can always make a 
unital algebra with an algebra norm into a unital normed algebra with the same topology.
  
\begin{definition}[Algebra homomorphism]
  Let \(A, B\) be algebras. A homomorphism from \(A\) to \(B\) is a linear map \(\theta : A \to B\) 
  such that 
  \[\theta(xy) = \theta(x) \theta(y)\]
  for all \(x, y \in A\).
  If \(A, B\) are in addition unital, then we also require \(\theta(1_A) = 1_B\).

  If \(\theta\) is bijective, then we say it is an isomorphism.
\end{definition}

We note that for \(A, B\) normed algebras, a homomorphism is \textit{not} assumed to be continuous 
while isomorphism is assumed to be continuous with a continuous inverse.

As our focus is on spectral theory, from this point forward, we will assume the scalar field is \(\mathbb{C}\).

\begin{example} 
  Let \(K\) be a compact Hausdorff space. Then \(C(K)\) is a commutative unital Banach algebra under 
  pointwise multiplication.

  Furthermore, a uniform algebra on \(K\) is a closed subalgebra of \(C(K)\) which separates points 
  of \(K\) and contain the constant functions. In the real case, Stone-Weierstrass implies that 
  it must be all of \(C(K)\). In our case however (with complex scalar field), Stone-Weierstrass in 
  addition requires the subalgebra to be closed under conjugation.

  An example of this is 
  \[A(\Delta) = \{f \in C(\Delta) \mid f \text{ holomorphic on } \Delta^\circ\}\]
  where \(\Delta = \{z \in \mathbb{C} \mid |z| \le 1\}.\)

  More generally, let \(K \subseteq \mathbb{C}\) be a non-empty compact subset. Then, we have the 
  following uniform algebras on \(K\):
  \[\mathcal{P}(K) \subseteq \mathcal{R}(K) \subseteq \mathcal{O}(K) \subseteq A(K) \subseteq C(K),\]
  where \(\mathcal{P}(K), \mathcal{R}(K), \mathcal{O}(K)\) are the closures of resp. 
  polynomials, rational functions without poles in \(K\) and holomorphic functions on some 
  open neighborhood of \(K\). We shall see later that \(\mathcal{R}(K) = \mathcal{O}(K)\) (always) and 
  \[\mathcal{P}(K) = \mathcal{R}(K) \iff \mathbb{C} \setminus K \text{ is connected.}\]
  On the other han, \(\mathcal{R}(K) \neq A(K)\) and 
  \[A(K) = C(K) \iff K^\circ = \varnothing.\]
\end{example}

\begin{example}
  \(L_1(\mathbb{R})\) with the \(L_1\)-norm and convolution as multiplication is a commutative 
  Banach algebra without a unit (Riemann-Lebesgue lemma).
\end{example}

\begin{example}
  Let \(X\) be a Banach space. Then \(\mathcal{B}(X)\) (bounded linear operators from \(X\) to itself)
  with the operator norm and composition as multiplication is a unital Banach algebra.
  It is not commutative if \(\dim X \ge 2\). 

  In the special case that \(X\) is a Hilbert space, then \(\mathcal{B}(X)\) is what is known as 
  a \(C^*\)-algebra.
\end{example}

\subsection{Constructions}

\textit{Subalgebra}: Let \(A\) be an algebra and \(B\) a subalgebra of \(A\). If \(A\) is unital with unit 
\(1\), then \(B\) is unital if \(1 \in B\). If \(A\) is a normed algebra, then \(\overline{B}\) 
is also a subalgebra.

\textit{Unitization}: If \(A\) is a normed algebra. The unitization of \(A\) is the vector space 
\(A_+ = A \oplus \mathbb{C}\) with multiplication 
\[(a, \lambda)(b, \mu) = (ab + \lambda a + b\mu, \lambda \mu).\]
Then, \(A_+\) is a unital algebra with the unit \((0, 1)\). The set \(\{(a, 0) \mid a \in A\}\) is 
an ideal of \(A_+\) and is isomorphic as an algebra to \(A\). We write 
\[A_+ = \{a + \lambda 1 \mid a \in A, \lambda \in \mathbb{C}\}.\]
If \(A\) is a normed algebra, then so is \(A_+\) with the norm 
\[\|a + \lambda 1\| = \|a\| + |\lambda|\]
and in this case, \(A\) is a closed ideal of \(A_+\). Furthermore, if \(A\) is a Banach algebra, 
so is \(A_+\).

\textit{Ideals}: Let \(A\) be a normed algebra. If \(J \trianglelefteq A\), then also 
\(\overline{J} \trianglelefteq A\). If \(J\) is a closed ideal of \(A\), then we can define \(A / J\) 
which is a normed algebra with the quotient norm. 

If \(A\) is in addition unital, and \(J\) is a proper ideal, then \(A / J\) is a unital normed algebra
with the unit \(1 + J\).

\textit{Completion}: Let \(A\) be a normed algebra and \(\tilde A\) be its completion. For \(a, b \in \tilde A\), 
by construction, we may choose sequences \((a_n), (b_n) \subseteq A\) such that \(a_n \to a\) and 
\(b_n \to b\). Then, defining 
\[ab = \lim_{n \to \infty} a_n b_n\]
where the right hand side exists and it's Cauchy, \(\tilde A\) is a Banach algebra which contains 
\(A\) as a dense subalgebra.

\textit{Operator algebra}: Let \(A\) be a unital Banach algebra. For each \(a \in A\), we define 
\[L_a : A \to A: x \mapsto ax.\]
\(L_a\) is clearly linear, and is bounded as \(\|ax\| \le \|a\|\|x\|\). The map 
\(a \mapsto L_a : A \to \mathcal{B}(A)\) is an isometric homomorphism. Thus, every Banach algebra 
is a closed subalgebra of \(\mathcal{B}(X)\) for some \(X\). 

\begin{lemma}
  Let \(A\) be a unital Banach algebra and let \(a \in A\). Then, \(a\) is invertible if \(\|a - 1\| < 1\).
  Furthermore, 
  \[\|a^{-1}\| \le \frac{1}{1 - \|1 - a\|}.\]
\end{lemma}
\begin{proof}
  Let \(h = 1 - a\) so \(a = 1 - h\), \(\|h\| < 1\) and \(\|h^n\| \le \|h\|^n\). Thus, 
  \(\sum_{n = 0}^\infty \|h^n\|\) converges in \(\mathbb{R}\) and so \(b := \sum_{n = 0}^\infty h^n\) 
  converges in \(A\) (as \(A\) is a Banach space). With this in mind, we observe 
  \[ab = (1 - h)\sum_{n = 0}^\infty h^n = \sum h^n - \sum h^{n + 1} = 1.\]
  Similarly \(ba = 1\) so \(a\) is invertible. Moreover, 
  \[\|a^{-1}\| \le \sum_{n = 0}^\infty \|h\|^n = \frac{1}{1 - \|h\|} = \frac{1}{1 - \|1 - a\|}\]
  as required.
\end{proof}

We introduce the notation 
\[G(A) = \{a \in A \mid a \text{ invertible}\}.\]

\begin{corollary}
  Let \(A\) be a unital Banach algebra, then 
  \begin{enumerate}
    \item\label{cor:inv_open} \(G(A)\) is open.
    \item \(x \mapsto x^{-1} : G(A) \to G(A)\) is continuous.
    \item\label{cor:conv_ninv} If \((x_n) \subseteq G(A)\) converges to \(x \not\in G(A)\), then \(\|x_n^{-1}\| \to \infty\).
    \item If \(x \in \partial G(A)\), then there exists a sequence \((z_n)\) with \(\|z_n\| = 1\) for all 
      \(n\) such that 
      \[z_n x \to 0 \text{ and } x z_n \to 0.\]
      It follows that \(x\) has no left or right inverse (even in any unital Banach algebra containing 
      \(A\) isometrically).
  \end{enumerate}
\end{corollary}
\begin{proof}\(\)\newline
  \begin{enumerate}
    \item Let \(x \in G(A), y \in A\). If \(\|y - x\| < \|x^{-1}\|^{-1}\), then 
    \[\|1 - x^{-1}y\| = \|x^{-1}(x - y)\| \le \|x^{-1}\| \|x - y\| < 1\]
    and so, \(x^{-1}y \in G(A)\) and hence also \(y \in G(A)\). 
    \item Fix \(x, y \in G(A)\), then
    \[\|y^{-1} - x^{-1}\| = \|y^{-1}(x - y)x^{-1}\| \le \|y^{-1}\|\|x^{-1}\|\|x - y\|.\]
    Then, if \(\|x - y\| < (2 \|x^{-1}\|)^{-1}\), we have 
    \[\|y^{-1}\| - \|x^{-1}\| \le \|y^{-1} - x^{-1}\| \le \frac{1}{2}\|y^{-1}\|\]
    implying \(\|y^{-1}\| \le 2\|x^{-1}\|\). Thus, 
    \[\|y^{-1} - x^{-1}\| \le \|y^{-1}\|\|x^{-1}\|\|x - y\| \le 2 \|x^{-1}\|^2 \|x - y\|\]
    which converges to 0 as \(y \to x\).
    \item From~\ref{cor:inv_open}, for all \(y \in A\) and \(\|y - x_n\| < \|x_n^{-1}\|^{-1}\), we have 
    \(y \in G(A)\). Hence, \(\|x_n - x\| \ge \|x_n^{-1}\|^{-1}\) implying \(\|x_n^{-1}\| \to \infty\)
    as claimed.
    \item Choose \((x_n)\) in \(G(A)\) such that \(x_n \to x\). Then, defining 
    \[z_n := \frac{x_n^{-1}}{\|x_n^{-1}\|},\]
    we have \(\|z_n\| = 1\) and 
    \[\|z_n x\| = \|z_n x + z_n(x - x_n)\| \le \frac{1}{\|x_n^{-1}\|} + \|z_n\| \|x - x_n\|.\]
    Now, as \(\|x_n^{-1}\|^{-1}\) converges to \(0\) by~\ref{cor:conv_ninv}, the right hand side 
    converges to 0 as \(n \to \infty\) allowing us to conclude.
  \end{enumerate}
\end{proof}

\subsection{Spectrum} 

\begin{definition}[Spectrum]
  Let \(A\) be an algebra and let \(x \in A\). We define the spectrum \(\sigma_A(x)\) of \(x\) to be
  \[\sigma_A(x) := \{\lambda \in \mathbb{C} \mid \lambda 1 - x \not\in G(A)\}\]
  if \(A\) is unital and 
  \[\sigma_A(x) := \sigma_{A_+}(x)\]
  if \(A\) is not unital.
\end{definition}

\begin{example}
  If \(A = M_n(\mathbb{C})\), then \(\sigma_A(x)\) is the set of eigenvalues of \(x\).
\end{example}

\begin{example}
  If \(A = C(K)\) for a compact Hausdorff \(K\), \(f \in A\), then \(\sigma_A(f) = f(K)\) since 
  \(g \in A\) is invertible if and only if \(0 \not\in g(K)\).
\end{example}

\begin{example}
  If \(X\) is a Banach space, \(A = \mathcal{B}(X)\), \(T \in A\). Then, 
  \[\sigma_A(T) = \{\lambda \in \mathbb{C} \mid \lambda \text{id} - T \text{ is not an isomorphism}\}.\]
\end{example}

\begin{theorem}
  Let \(A\) be a Banach algebra and \(x \in A\). Then \(\sigma_A(x)\) is non-empty compact subset of 
  \(\{\lambda \in \mathbb{C} \mid |\lambda| \le \|x\|\}\). 
\end{theorem}
\begin{proof}
  By unitization, we may assume \(A\) is unital. Consider that the map 
  \[\lambda \mapsto \lambda 1 - x : \mathbb{C} \to A\]
  is continuous and \(\sigma_A(x)\) is the inverse image of \(A \setminus G(A)\) under this map, 
  \(\sigma_A(x)\) must be closed. 

  Now, if \(|\lambda| > \|x\|\), then \(\|x / \lambda\| < 1\) and so by the previous theorem, 
  \(1 - x / \lambda \in G(A)\). Thus, as \(\lambda \ne 0\), \(\lambda(1 - x / \lambda) = \lambda 1 - x \in G(A)\)
  and hence, \(\lambda \not \in \sigma_A(x)\). As we've shown that \(\sigma_A(x) \subseteq \mathbb{C}\) 
  is closed and bounded, it is thusly compact.

  Finally, we will show it is non-empty. Suppose otherwise, then we can define the (resolvent) map 
  \[R : \mathbb{C} \to G(A) : \lambda \mapsto (\lambda 1 - x)^{-1}\]
  which in particular is holomorphic since
  \[R(\lambda) - R(\mu) = R(\lambda)((\mu 1 - x) - (\lambda 1 - x))R(\mu)  = (\mu - \lambda)R(\lambda) - R(\mu).\]
  Thus, 
  \[\frac{R(\lambda) - R(\mu)}{\lambda - \mu} = -R(\lambda)R(\mu) \to -R(\mu)^2\]
  as \(\lambda \to \mu\) since \(R\) is continuous.

  Now, for \(|\lambda| > \|x\|\), \(R(\lambda) = \lambda^{-1}(1 - x / \lambda)^{-1}\) 
  and so, 
  \[\|R(\lambda)\| \le \frac{1}{|\lambda|} \frac{1}{1 - \|x / \lambda\|} = \frac{1}{|\lambda| - \|x\|}\]
  which converges to \(0\) as \(|\lambda| \to \infty\). Hence, \(R = 0\) by the vector valued Liouville's theorem 
  which is a contradiction.
\end{proof}

\begin{corollary}[Gelfand-Mazur]
  A complex unital normed division algebra (i.e. \(G(A) = A \setminus \{0\}\)) is isometrically isomorphic
  to \(C\).
\end{corollary}
\begin{proof}
  The map we want is
  \[\theta : \mathbb{C} \to A : \lambda \mapsto \lambda 1.\]
  It is clear that \(\theta\) is an isometric homomorphism. 
  
  For surjectivity, let \(B\) be a completion of \(A\), so \(B\) is a unital Banach algebra. Given 
  \(x \in A\), by the previous theorem \(\sigma_B(x)\) is non-empty and so we may choose \(\lambda \in \sigma_B(x)\).
  Then, \(\lambda 1 - x \not \in G(B)\) and so \(\lambda 1 - x \not \in G(A)\). However, as \(A\) is a 
  division algebra, this means \(\lambda 1 - x = 0\) and so \(\theta(\lambda) = x\) as required.
\end{proof}

\begin{definition}[Spectral radius]
  Let \(A\) be a Banach algebra and \(x \in A\). The spectral radius of \(x\) is defined to be 
  \[r_A(x) := \sup \{|\lambda| \mid \lambda \in \sigma_A(x)\} \le \|x\|.\]
\end{definition}

\begin{lemma}
  If \(A\) is a unital algebra, \(x, y \in A\) and \(xy = yx\), then 
  \(x, y \in G(A)\) if and only if \(xy \in G(A)\).
\end{lemma}
\begin{proof}
  Let \(b = (xy)^{-1}\), then, \((by)x = b(yx) = b(xy) = 1 = (xy)b = x(yb)\).
\end{proof}

\begin{lemma}[Polynomial spectral mapping theorem]
  Let \(A\) be a unital Banach algebra and let \(x \in A\). Then, for any complex polynomial 
  \(p(x) = \sum_{k = 0}^n a_k z^k\), we have 
  \[\sigma_A(p(x)) = \{p(\lambda) \mid \lambda \in \sigma_A(x)\}.\] 
\end{lemma}
\begin{proof}
  The lemma is clear for constant polynomials as \(\sigma_A(\lambda 1) = \{\lambda\}\).

  Assume now \(n \ge 1\) and \(a_n \neq 0\). Then, fixing \(\mu \in \mathbb{C}\), we write 
  \[\mu - p(z) = c \prod_{\gamma = 1}^n (\lambda_\gamma - z)\]
  for some \(c \neq 0\), \(\lambda_1, \cdots, \lambda_n \in \mathbb{C}\). Then, by the above lemma,
  \[\mu 1 - p(x) = c \prod_{\gamma = 1}^n (\lambda_\gamma 1 - x)\]
  is invertible if and only if \(\lambda_\gamma 1 - x\) is invertible for all \(\gamma\). Thus, 
  \(\mu \in \sigma_A(p(x))\) if and only if one of the \(\lambda_\gamma \in \sigma_A(x)\) which 
  occurs if and only if \(p(\lambda_\gamma) = \mu\).
\end{proof}

\begin{theorem}[Beurling-Gelfand spectral radius formula]
  Let \(A\) be a Banach algebra and let \(x \in A\). Then, 
  \[r_A(x) = \lim_{n \to \infty}\|x^n\|^{1 / n} = \inf_n \|x^n\|^{1 / n}.\]
\end{theorem}
\begin{proof}
  By unitization, we may assume \(A\) is unital. 

  Observe that for \(\lambda \in \sigma_A(x)\), \(\lambda^n \in \sigma_A(x^n)\) (by polynomial 
  spectral mapping) and so \(|\lambda^n| \le \|x^n\|\). Thus, \(|\lambda| \le \|x^n\|^{1 / n}\) and 
  it follows that \(r_A(x) \le \inf_n \|x^n\|^{1 / n}\).

  Consider again the resolvent operator
  \[R : \{\lambda \in \mathbb{C} \mid |\lambda| > r_A(x)\} \to G(A) : \lambda \to (\lambda 1 - x)^{-1}.\]
  We've previously shown \(R\) is holomorphic and hence, for any \(\phi \in A^*\), \(\phi \circ R\) 
  has a Laurent expansion. In particular, for \(|\lambda| > \|x\| (\ge r_A(x))\), we have 
  \[R(\lambda) = \frac{1}{\lambda}\left(1 - \frac{x}{\lambda}\right)^{-1} 
    = \frac{1}{\lambda}\sum_{n = 0}^\infty \frac{x^n}{\lambda^n}.\]
  Hence, 
  \[\phi \circ R(\lambda) = \frac{1}{\lambda} \sum_{n = 0}^\infty \phi\left(\frac{x^n}{\lambda^n}\right)
    = \sum_{n = 0}^\infty \frac{\phi(x^n)}{\lambda^{n + 1}}\]
  implying \(\sum_{n = 0}^\infty \frac{\phi(x^n)}{\lambda^{n + 1}}\) is the Laurent expansion of \(\phi \circ R\).
  Thus, for all \(\lambda \in \mathbb{R}\) with \(|\lambda| > r_A(x)\), \(\phi(x^n / \lambda^n) \to 0\) for
  any \(\phi \in A^*\). With this, \(\{x^n / \lambda^n \mid n \in \mathbb{N}\}\) is weakly bounded and hence 
  is bounded in norm by some constant \(M\). Then, for all \(n\), \(\|x^n / \lambda^n\| \le M\) and so, 
  \[\|x^n\|^{1 / n} \le M^{1 / n}|\lambda| \text{ implying } \limsup \|x^n\|^{1 / n} \le |\lambda|\]
  for every \(\lambda\) satisfying \(|\lambda| > r_A(x)\). Thus, we have 
  \[r_A(x) \le \inf \|x^n\|^{1 / n} \le \liminf \|x^n\|^{1 / n} \le \limsup \|x^n\|^{1 / n} \le r_A(x).\] 
\end{proof}

\begin{theorem}\label{thm:spectrum_subalg}
  Let \(A\) be a unital Banach algebra and \(B\) a unital subalgebra of \(A\). Then, given \(x \in B\), 
  \[\sigma_B(x) \supseteq \sigma_A(x) \text{ and } \partial \sigma_B(x) \subseteq \partial \sigma_A(x).\]
  It follows that \(\sigma_B(x)\) is the union of \(\sigma_A(x)\) with some of the bounded components 
  of \(\mathbb{C} \setminus \sigma_A(x)\).
\end{theorem}
\begin{proof}
  If \(\lambda \not\in \sigma_B(x)\), then \(\lambda 1 - x \in G(B)\) and so, \(\lambda 1 - x \in G(A)\) 
  implying \(\lambda \not\in \sigma_A(x)\). 

  On the other hand, let us take \(\lambda \in \partial \sigma_B(x)\) (\(\lambda \in \sigma_B(x)\) as 
  \(\sigma_B(x)\) is compact and hence closed). So, choosing 
  \((\lambda_n) \subseteq \mathbb{C} \setminus \sigma_B(x) \subseteq \mathbb{C} \setminus \sigma_A(x)\) such that \(\lambda_n \to \lambda\), 
  it suffices to show that \(\lambda \in \sigma_A(x)\).

  Observe that \(\lambda_n 1 - x \in G(B) \subseteq G(A)\) for all \(n\) and 
  \(\lambda_n 1 - x \to \lambda 1 - x \not\in G(B)\). Namely, \(\lambda 1 - x \in \partial G(B)\). 
  Thus, if \(\lambda 1 - x \in G(A)\), by the continuity of the inverse, 
  \[(\lambda_n 1 - x)^{-1} \to (\lambda 1 - x)^{-1}.\]
  However, as \((\lambda_n 1 - x)^{-1} \in B\), and \(B\) is closed, it follows \((\lambda 1 - x)^{-1} \in B\) 
  contradicting \(\lambda 1 - x \not\in G(B)\). Hence, \(\lambda 1 - x \not\in G(A)\) implying 
  \(\lambda \in \sigma_A(x)\) as required. 
\end{proof}

\begin{proposition}
  Let \(A\) be a unital Banach algebra and \(C\) a maximal commutative subalgebra of \(A\). Then \(C\)
  is closed, unital and for all \(x \in C\), we have \(\sigma_A(x) = \sigma_C(x)\).  
\end{proposition}
\begin{proof}
  As multiplication is continuous, it follows \(\overline{C}\) is also a commutative subalgebra. Thus, 
  for \(C\) to be maximal, \(C = \overline{C}\) implying \(C\) is closed. \(C\) is unital as 1 commutes 
  with all elements of \(C\) and so can always be added in to create a larger commutative subalgebra.

  Fix \(x \in C\). We already know \(\sigma_C(x) \supseteq \sigma_A(x)\). Now, for \(\lambda \not\in \sigma_A(x)\), 
  there exists some \(y \in A\), 
  \[y(\lambda 1 - x) = (\lambda 1 - x)y = 1.\]
  On the other hand, as \(\lambda 1 - x \in C\), it commutes with any \(z \in C\). Thus, 
  \[yz = yz(\lambda 1 - x)y = y(\lambda 1 - x)zy = zy\]
  implying \(y \in C\) by maximality. Thus \(\lambda \not\in \sigma_C(x)\) as required.
\end{proof}

\subsection{Commutative Banach algebra}

\begin{definition}[Character]
  A character on an algebra \(A\) is a non-zero homomorphism \(\phi : A \to \mathbb{C}\). We denote the 
  set of all characters on \(A\) by \(\Phi_A\) and we call it the spectrum of \(A\) (when it is 
  equipped with the Gelfand topology, see below).
\end{definition}

In the case \(A\) is unital, then for all \(\phi \in \Phi_A\), \(\phi(1) = 1\).

\begin{lemma}
  Let \(A\) is a Banach algebra, \(\phi \in \Phi_A\), then \(\phi\) is bounded and \(\|\phi\| \le 1\).
  Moreover, if \(A\) is unital, then \(\|\phi\| = 1\).
\end{lemma}
\begin{proof}
  By defining \(\phi_+ : A_+ \to \mathbb{C}\), \(\phi_+(x + \lambda 1) = \phi(x) + \lambda\), we have 
  \(\phi_+ \in \Phi_+\) with \(\phi_+|_{A} = \phi\). Thus, it suffices to show \(\|\phi_+\| \le 1\) 
  allowing us to assume \(A\) is unital. 

  Let \(x \in A\) and suppose \(\phi(x) > \|x\|\). Then, \(\phi(x)1 - x \in G(A)\) 
  (since for all \(\lambda \in \sigma_A(x)\), \(|\lambda| \le \|x\|\)). Thus, there exists some \(y \in A\) 
  such that \((\phi(x)1 - x)y = 1\) and applying \(\phi\) on both sides results in
  \[1 = \phi(1) = (\phi(\phi(x) 1) - \phi(x))\phi(y) = (\phi(x) - \phi(x))\phi(y) = 0\]
  which is a contradiction. Thus, \(\phi(x) \le \|x\|\). On the other hand, as \(\|\phi(1)\| = 1\), 
  it follows \(\|\phi\| = 1\).
\end{proof}

\begin{lemma}
  Let \(A\) be a unital Banach algebra. If \(J\) is a proper ideal of \(A\), then so is \(\overline{J}\). 
  Hence, maximal ideals are always closed.
\end{lemma}
\begin{proof}
  Since \(J\) is proper, \(J \cap G(A) = \varnothing\). Thus, as \(G(A)\) is open, we also have 
  \(\overline{J} \cap G(A) = \varnothing\). Hence, \(\overline{J}\) is a proper ideal of \(A\) 
  as required.
\end{proof}

We introduce the notation \(\mathcal{M}_A\) for the set of all maximal ideals of \(A\).

\begin{theorem}
  Let \(A\) be a commutative unital Banach algebra. Then the map 
  \[\phi \mapsto \ker \phi : \Phi_A \to \mathcal{M}_A\]
  is a bijection.
\end{theorem}
\begin{proof}
  Firstly, the map is well-defined as it is clear \(\ker \phi\) is an ideal of \(A\) while 
  it is maximal since \(\text{codim}(\phi) = 1\).

  \textit{Injectivity}: Let \(\phi, \psi \in \Phi_A\) with \(\ker \phi = \ker \psi\). Then, for all 
  \(x \in A\), \(\phi(x)1 - x \in \ker \psi\) and thus, \(\phi(x) - \psi(x) = 0\) as required.

  \textit{Surjectivity}: Let \(M \in \mathcal{M}_A\) so \(A / M\) is a field and a unital Banach algebra.
  By Gelfand-Mazur, \(A / M\) is isometrically isomorphic to \(\mathbb{C}\) and thus the quotient 
  map is a character with kernel \(M\)
\end{proof}

\begin{corollary}
  Let \(A\) be a commutative unital Banach algebra with \(x \in A\). Then,
  \begin{itemize}
    \item \(x \in G(A)\) if and only if for all \(\phi \in \Phi_A\), \(\phi(x) \ne 0\).
    \item \(\sigma_A(x) = \{\phi(x) \mid \phi \in \Phi_A\}\).
    \item \(r_A(x) = \sup \{|\phi(x)| \mid \phi \in \Phi_A\}\).
  \end{itemize}
\end{corollary}
\begin{proof}\(\)\newline
  \begin{itemize}
    \item If \(x \in G(A)\), then for all \(\phi \in \Phi_A\), 
      \(1 = \phi(1) = \phi(x x^{-1}) = \phi(x)\phi(x^{-1}) = 0\) if \(\phi(x) = 0\).

      On the other hand, if \(x \not\in G(A)\), we can define \(M\) to be a maximal ideal containing 
      \(x\). Thus, by the above theorem, there exists some \(\phi \in \Phi_A\) such that \(\ker \phi = M \ni x\).
    \item By the first part, \(\lambda \in \sigma_A(x)\) if and only if there exists some \(\phi \in \Phi_A\)
      such that \(\phi(\lambda 1 - x) = 0\). Namely \(\lambda \in \sigma_A(x)\) if and only if there exists 
      some \(\phi \in \Phi_A\) such that \(\phi(x) = \lambda\).
    \item Clear by the second part.
  \end{itemize}
\end{proof}

\begin{corollary}\label{cor:commut_radius}
  Let \(A\) be a Banach algebra and \(x, y \in A\) such that \(xy = yx\). Then, 
  \[r_A(x + y) \le r_A(x) + r_A(y) \text{ and } r_A(xy) \le r_A(x) r_A(y).\]
\end{corollary}
\begin{proof}
  By unitization we may assume \(A\) is unital. By consider the subalgebra generated by \(x, y\), 
  we can also assume \(A\) is commutative. Thus, the conclusion is clear by the third part of the above corollary.
\end{proof}

\begin{example}
  Let \(A = C(K)\) where \(K\) is a compact Hausdorff space. Then 
  \[\Phi_A = \{\delta_k \mid k \in K\}\]
  where \(\delta_k(f) = f(k)\).

  Clearly \(\delta_k \in \Phi_A\) for any \(k\) so we will only consider the reverse inclusion.
  Let \(M \in \mathcal{M}_A\) and we need to show that there is some \(k \in K\) such that 
  \[M = \{f \mid f(k) = 0\}.\]
  Suppose otherwise, then for all \(k \in K\), there exists some \(f_k \in M\) such that 
  \(f_k(k) \ne 0\). Then, for each \(k\), \(f_k\) is not zero on a open neighborhood of \(k\). Let 
  us denote this neighborhood by \(U_k\) and it is clear that \(\{U_k\}_{k \in K}\) form an open 
  cover of \(K\). Thus, by compactness, there exists \(k_1, \cdots, k_n\) such that 
  \(\{U_{k_i}\}_{i = 1}^n\) covers \(K\). Now, defining 
  \[g := \sum_{i = 1}^n |f_{k_i}|^2,\]
  by construction, \(g\) does not have any 0 in \(K\) and in particular, \(g \in G(A)\). However, 
  since \(g = \sum_{i = 1}^n \overline{f_{k_i}} f_{k_i} \in M\), it follows that \(M = A\) which is 
  a contradiction.
\end{example}

\begin{example}
  Let \(K \subseteq \mathbb{C}\) be a non-empty compact set. Then \(\Phi_{C(K)} = \{\delta_\omega \mid \omega \in K\}\).
\end{example}

\begin{example}
  For \(A(\Delta)\) the disk algebra, \(\Phi_{A(\Delta)} = \{\delta_n \mid n \in \Delta\}\).
\end{example}

\begin{example}
  Denote the Wiener algebra 
  \[W = \left\{f \in C(\mathbb{T}) \mid \sum_{n \in \mathbb{Z}} |\hat{f}(n)| < \infty\right\}\]
  where \(\mathbb{T} = \{z \in \mathbb{C} \mid |z| = 1\}\), 
  \(\hat{f}(n) = (2\pi)^{-1}\int_0^{2\pi} f(e^{i\theta}) e^{-in\theta} d\theta\). Then, \(W\) is a commutative 
  unital Banach algebra with pointwise operations and norm 
  \[\|f\|_1 = \sum_{n \in \mathbb{Z}} |\hat{f}(n)|.\]
  This is isometrically isomorphic to the commutative unital Banach algebra \(l_1(\mathbb{Z})\) with 
  pointwise vector operations, the \(l_1\)-norm and convolution as multiplication: 
  \[(a * b)_n = \sum_{j + k = n} a_j b_k.\]
  In this case, \(\Phi_W  = \{\delta_\omega \mid \omega \in \mathbb{T}\}\) and so \(f \in W\) is 
  invertible if and only if \(f\) is nowhere 0 (Wiener's theorem). 
\end{example}

Let \(A\) be a commutative unital Banach algebra. Then, we may write 
\begin{align*}
  \Phi_A & = \{\phi \in B_{A^*} \mid \phi(1) = 1, \phi(xy) = \phi(x)\phi(y),\ \forall x, y \in A\}\\
    & = B_{A^*} \cap \hat{1}^{-1}(1) \cap \bigcap_{x, y \in A} (\hat{xy} - \hat{x} \hat{y})^{-1}(0).
\end{align*}
which is a w\(^*\)-closed subset of \(A\). So, \(\Phi_A\) is a \textit{compact} Hausdorff space in the w\(^*\)-topology
and this topology is known as the Gelfand topology. We call \(\Phi_A\) equipped with the Gelfand 
topology the spectrum of \(A\).

For \(x \in A\), we define its Gelfand transform to be 
\[\hat x : \Phi_A \to \mathbb{C} : \phi \mapsto \phi(x).\]
The map \(x \mapsto \hat x : A \to C(\Phi_A)\) is called the Gelfand map. 

\begin{theorem}[Gelfand representation theorem]
  The Gelfand map \(A \to C(\Phi_A)\) is a continuous, unital homomorphism. Moreover, for \(x \in A\), 
  \begin{itemize}
    \item \(\|\hat x\|_\infty = r_A(x) \le \|x\|\).
    \item \(\sigma_{C(\Phi_A)}(\hat x) = \sigma_A(x)\).
    \item \(x \in G(A)\) if and only if \(\hat x \in G(C(\Phi_A))\).
  \end{itemize}
\end{theorem}
\begin{proof}
  Continuity and the first part of the theorem follows as 
  \[\|\hat x\| = \sup \{|\hat x(\phi)| \mid \phi \in \Phi_A\} = \sup \{\phi(x) \mid \phi \in \Phi_A\} = r_A(x) \le \|x\|.\]
  The second part of the theorem follows as
  \[\sigma_{C(\Phi_A)}(\hat x) = \{\phi(x) \mid \phi \in \Phi_A\} = \sigma_A(x)\]
  where the first equality holds since by the above example, \(\Phi_{C(\Phi_A)} = \{\delta_\phi \mid \phi \in \Phi_A\}\).

  Finally, the third part follows directly from the second.
\end{proof}

We remark that the Gelfand map is in general \textit{not} injective nor surjective. The Gelfand map has 
kernel 
\[\{x \in A \mid r_A(x) = 0\} = \{x \mid \liminf_{n \to \infty} \|x^n\|^{1 / n} = 0\} 
  = \bigcap_{\phi \in \Phi_A} \ker \phi = \bigcap_{M \in \mathcal{M}_A} M.\] 
We call \(x \in A\) quasi-nilpotent if \(\liminf_{n \to \infty} \|x^n\|^{1 / n} = 0\). 
The ideal \(J(A) := \bigcap_{M \in \mathcal{M}_A} M\) is known as the Jacobson radical and we say 
\(A\) is semi-simple if \(J(A) = 0\).

\newpage
\section{Holomorphic Functional Calculus}

Let \(U \subseteq \mathbb{C}\) be non-empty and open. Recall that 
\[\mathcal{O}(U) := \{f : U \to \mathbb{C} \mid f \text{ holomorphic}\}\]
which is a locally convex space with seminorms 
\[\|f\|_K = \sup_{x \in K} |f(x)|\]
for all \(K \subseteq U\) non-empty and compact.

\(\mathcal{O}(U)\) is an algebra with pointwise multiplication.

We introduce the notation \(e, u \in \mathcal{O}(U)\) where \(e(z) = 1, u(z) = z\) 
for all \(z \in U\). \(\mathcal{O}(U)\) is then unital with the unit \(e\).

The main theorem of the chapter is the following.

\begin{theorem}[Holomorphic functional calculus, HFC]
  Let \(A\) be a commutative Banach algebra with \(x \in A\). Let \(U \subseteq \mathbb{C}\) be 
  a non-empty open set with \(\sigma_A(x) \subseteq U\). Then, there exists a unique, continuous, 
  unital homomorphism 
  \[\Theta_x : \mathcal{O}(U) \to A, \text{ such that } \Theta_x(u) = x.\]
  Moreover, for all \(\phi \in \Phi_A, f \in \mathcal{O}(U)\) we have \(\phi(\Theta_x(f)) = f(\phi(x))\)
  and 
  \[\sigma_A(\Theta_x(f)) = \{f(\lambda) \mid \lambda \in \sigma_A(x)\}.\]
\end{theorem}

Heuristically, we can think of \(\Theta_x\) as the evaluation at \(x\) and we write \(f(x)\) for 
\(\Theta_x(f)\).

Since \(e(x) = \Theta_x(e) = 1\), \(u(x) = \Theta_x(u) = x\) and \(\Theta_x\) is a homomorphism, 
it follows that if \(p(z) = \sum_{k = 0}^n a_k z^k\) is a complex polynomial, then 
\(p(x) = \Theta_x(p) = \sum_{k = 0}^n a_k x^k\).

To prove this theorem, we will need Runge's approximation theorem which allows us to approximate
holomorphic functions by rational functions. 

\begin{theorem}[Runge's approximation theorem]
  Let \(K \subseteq \mathbb{C}\) be a non-empty compact set. Then, \(\mathcal{O}(K) = \mathcal{R}(K)\), 
  i.e. if \(f\) is holomorphic on some open set containing \(K\), then for all \(\epsilon > 0\), 
  there exists a rational function \(r\) without poles in \(K\) such that \(\|f - r\|_K < \epsilon\). 

  More precisely, given a set \(\Lambda\) which contains a point from each bounded component of 
  \(\mathbb{C} \setminus K\). For any \(\epsilon > 0\) and \(f\) holomorphic on some open set 
  containing \(K\), there exists a rational function \(r\) with poles in \(\Lambda\) such that 
  \(\|f - r\|_K < \epsilon\).
\end{theorem}

We remark that if \(\mathbb{C} \setminus K\) is connected, then taking \(\Lambda = \varnothing\), we 
have \(\mathcal{O}(K) = \mathcal{P}(K)\).

\subsection{Vector-valued integration}

Let \(a < b\) be in \(\mathbb{R}\), \(X\) a Banach space, \(f : [a, b] \to X\) a continuous function.
We define the integral of \(f\) over \([a, b]\) as follows:

Take sequences \(\mathcal{D}_n\): \(a = t_0^{(n)} < t_1^{(n)} < \cdots < t_{k_n}^{(n)} = b\)
such that \(|\mathcal{D}_n| =  \max_{1 \le j \le k_n} |t_j^{(n)} - t_{j - 1}^{(n)}| \to 0\) as 
\(n \to \infty\).

Since \(f\) is continuous on a compact set, it is uniformly continuous and so the limit 
\[\lim_{n \to \infty} \sum_{j = 1}^{k_n}f(t_j^{(n)})(t_j^{(n)} - t_{j - 1}^{(n)})\]
exists and is independent of the choice of \(\mathcal{D}_n\). We denote this limit by 
\(\int_a^b f(t) \dd t\).

We note that for \(\phi \in X^*\), 
\[\phi\left(\int_a^b f(t) \dd t\right) = \int_a^b \phi(f(t)) \dd t.\]
If \(\phi\) is the norming functional of \(\int_a^b f(t) \dd t\), then 
\[\left\|\int_a^b f(t) \dd t\right\| = \phi\left(\int_a^b f(t) \dd t\right) = 
  \int_a^b \phi(f(t)) \dd t \le \int_a^b \|\phi\|\|f(t)\| \dd t = \int \|f(t)\| \dd t.\]

Next, let \(\gamma : [a, b] \to \mathbb{C}\) be a path (i.e. \(\gamma\) continuously differentiable) 
and \(f : [\gamma] \to X\) a continuous function (\([\gamma] = \gamma([a, b])\)). Define 
\[\int_\gamma f(z) \dd z = \int_a^b f(\gamma(t)) \gamma'(t) \dd t.\]
For a chain \(\Gamma = (\gamma_1, \cdots, \gamma_n)\) and \(f : [\Gamma] \to X\) (where 
\([\Gamma] = \bigcup_{i = 1}^n [\gamma_i]\)), we define 
\[\int_\Gamma f(z) \dd z = \sum_{i = 1}^n \int_{\gamma_i} f(z) \dd z.\]
From this, we observe 
\[\left\|\int_\Gamma f(z) \dd z \right\| \le \sum_{i = 1}^n \left\|\int_{\gamma_i} f(z) \dd z\right\|
  \le l(\Gamma) \sup_{z \in \Gamma} \|f(z)\|\]
where \(l(\Gamma)\) denotes the length of \(\Gamma\), i.e. \(l(\Gamma) = \sum_{i = 1}^n l(\gamma_i)\)
where \(l(\gamma) = \int_a^b |\gamma'(t)| \dd t\) for any path \(\gamma\).

\subsection{Proof of HFC}

\begin{theorem}[Vector valued Cauchy theorem]
  Let \(U \subseteq \mathbb{C}\) be an open set, \(\Gamma\) a cycle in \(U\) such that 
  \(n(\Gamma, \omega) = 0\) for all \(\omega \not\in U\). Then, for a holomorphic function 
  \(f : U \to X\), we have 
  \[\int_\Gamma f(z) \dd z = 0.\]
\end{theorem}
\begin{proof}
  Indeed, for all \(\phi \in X^*\), 
  \[\phi\left(\int_\Gamma f(z) \dd z\right) = \int_\Gamma \phi(f(z)) \dd z = 0\]
  as \(\phi \circ f\) is holomorphic. Hence, the result follows by Hahn-Banach.
\end{proof}

\begin{lemma}
  Let \(A\) be a unital Banach algebra and \(x \in U \subseteq \mathbb{C}\) for some non-empty open 
  set \(U\). Furthermore, denote \(K = \sigma_A(x)\). Then, for a cycle \(\Gamma\) in \(U \setminus K\), 
  with 
  \[n(\Gamma, \omega) =
    \begin{cases}
    1, & \omega \in K,\\
    0, & \omega \not\in K,
  \end{cases}\]
  defining 
  \[\Theta_x : \mathcal{O}(U) \to A,\ f \mapsto \frac{1}{2\pi i} \int_\Gamma f(z)(z1 - x)^{-1} \dd z,\]
  then \(\Theta_x\) is well-defined, unital, linear and continuous. Furthermore, 
  \begin{itemize}
    \item For a rational function \(r\) without poles in \(U\), we have \(\Theta_x(r) = r(x)\). 
    \item For all \(\phi \in \Phi_A\) and \(f \in \mathcal{O}(Y)\), 
      \[\phi(\Theta_x(f)) = f(\phi(x)) \text{ and } \sigma_A(\Theta_x(f)) = f(\sigma_A(x)).\] 
  \end{itemize}
\end{lemma}

We remark that the above lemma is not quite HFS. Indeed, it is missing the condition that \(\Theta_x\) 
is multiplicative.

\begin{proof}
  \textit{Well-defined}: By construction, as \(\Gamma \cap K = \varnothing\), for all \(z \in [\Gamma]\), 
  \(z1 - x \in G(A)\) and so \(f(z)(z1 - x)^{-1}\) is defined on \([\Gamma]\) and furthermore is continuous.

  \textit{Linearity}: follows as the integral is linear.

  \textit{Continuity}: 
  \[\|\Theta_x(f)\| \le \frac{1}{2\pi} l(\Gamma) \sup_{z \in [\Gamma]} |f(z)| \|(z1 - x)^{-1}\|
    = M\sup_{z \in [\Gamma]} |f(z)| = M \|f\|_{[\Gamma]}\]
  with \(\|\cdot\|_{[\Gamma]}\) being one of the semi-norms used to define the LCS topology on 
  \(\mathcal{O}(U)\) implying \(\Theta_x\) is continuous.

  \textit{Unital}: 
  \[\Theta_x(e) = \frac{1}{2\pi i} \int_\Gamma (z1 - x)^{-1} \dd z 
      = \frac{1}{2\pi i} \int_{|z| = R} (z1 - x)^{-1} \dd x
      = \frac{1}{2\pi i} z^{-1} \int_{|z| = R} (1 - x / z)^{-1} \dd z\]
  where the second equality holds as \(\Gamma\) and \(\{|z| = R\}\) are homologous in 
  \(\mathbb{C} \setminus K\) for sufficiently large \(R\). Thus, taking \(R > \|x\|\), we have 
  \[\Theta_x(e)  
    = \frac{1}{2\pi i} \int_{|z| = R} \sum_{n = 0}^\infty \frac{x^n}{z^{n + 1}} \dd z 
    = \sum_{n = 0}^\infty \left(\frac{1}{2\pi i} \int_{|z| = R} \frac{\dd z}{z^{n + 1}}\right) x^n = 1\]
  
  Now, given rational function \(r\) without poles in \(U\), we can write \(r = p / q \in \mathcal{O}(U)\) 
  where \(p, q\) are complex polynomials where \(q\) does not have any zeros in \(U\). By the polynomial 
  spectral mapping theorem, we know that 
  \[\sigma_A(q(x)) = \{q(\lambda) \mid \lambda \in \sigma_A(x)\}.\]
  Thus, \(0 \not\in \sigma_A(q(x))\) implying \(q(x) \in G(A)\) and we may define \(r(x) = p(x) q(x)^{-1}\).
  Denote \(s\) the complex polynomial in two variables such that for all \(z, w \in \mathbb{C}\), 
  we have
  \[p(z)q(w) - q(z)p(w) = (z - w)s(z, w).\]
  Hence, 
  \[p(z)q(x) - q(z)p(x) = (z1 - x)s(z, x)\]
  implying 
  \[r(z)1(z1 - x)^{-1} - r(x)(z1 - x)^{-1} = s(z, x)q(z)^{-1}q(x)^{-1}.\]
  Thus, 
  \[\Theta_x(r) = \frac{1}{2\pi i} \int_\Gamma r(z)(z1 - x)^{-1} \dd z 
    = \frac{1}{2\pi i}\int_\Gamma (z1 - x)^{-1} \dd z r(x) 
      + \frac{1}{2\pi i} \int_\Gamma s(z, x)q(z)^{-1} \dd z q(x)^{-1}\]
  where \(\frac{1}{2\pi i}\int_\Gamma (z1 - x)^{-1} \dd z = \Theta_x(e) = 1\) and the second term 
  vanishes by the scalar Cauchy integral formula resulting in \(\Theta_x(r) = r(x)\) as required.

  Finally, fixing \(\phi \in \Phi_A\) and \(f \in \mathcal{O}(U)\), we have
  \begin{align*}
    \phi(\Theta_x(f)) & = \phi \left(\frac{1}{2\pi i}\int_\Gamma f(z)(z1 - x)^{-1} \dd z\right) 
        = \frac{1}{2\pi i} \int_\Gamma \frac{f(z)}{z - \phi(x)} \dd z\\
      & = n(\Gamma, \phi(x)) f(\phi(x)) = f(\phi(x))
  \end{align*}
  where \(n(\Gamma, \phi(x)) = 1\) since \(\phi(x) \in \sigma_A(x) = K\). With this, we have
  \[\sigma_A(\Theta_x(f)) = \{\phi(\Theta_x(f)) \mid \phi \in \Phi_A\}
    = \{f(\phi(x)) \mid \phi \in \Phi_A\} = \{f(\lambda) \mid \lambda \in \sigma_A(x)\}\]
  as required.
\end{proof}

\begin{proof}[Proof of Rangu's theorem]
  Let \(U \subseteq \mathbb{C}\) be open such that \(U \supseteq K\). Let \(A = \mathcal{R}(K)\), 
  \(x, \in A\) be such that \(x(z) = z\) for all \(z \in K\). Then, 
  \[\sigma_A(x) = \{\phi(x) \mid \phi \in \Phi_A\} = \{\delta_z(x) \mid z \in K\} = K.\]
  Then, for all \(f \in \mathcal{O}(U), z \in K\), we have 
  \(\Theta_x(f) = \delta_z(\Theta_x(f)) = f(\delta_z(x)) = f(z)\), namely \(\Theta_x(f) = f|_K\). 
  Next, taking \(B\) to be the closed subalgebra of \(A\) generated by \(1, x, (\lambda 1 - x)^{-1}\), 
  for all \(\lambda \in \Lambda\), i.e. \(B\) is the 
  closure in \(C(K)\) of rational functions whose poles are in \(\Lambda\) 
  (recall that \(\Lambda \subseteq \mathbb{C}\) is a given set which 
  contains a point from each bounded components of \(\mathbb{C} \setminus K\)). It is clear that 
  \(B\) is a closed unital subalgebra of \(A\). 

  If \(V\) is a bounded component of \(\mathbb{C} \setminus K = \mathbb{C} \setminus \sigma_A(x)\), 
  then there exists some \(\lambda \in \Lambda \cap V\) such that \(\lambda 1 - x\) is invertible in 
  \(B\), namely \(\lambda \not\in \sigma_B(x)\). It follows then \(\sigma_B(x) = \sigma_A(x) = K\). 
  Hence, \(\Theta_x\) takes value in \(B\). 
\end{proof}

\begin{corollary}
  Let \(U \subseteq \mathbb{C}\) be a non-empty open set. Then, the subalgebra rational functions of 
  \(\mathcal{O}(U)\) is dense, i.e. \(\mathcal{R}(U) = \mathcal{O}(U)\).
\end{corollary}
\begin{proof}
  Let \(K \subseteq U\) be a non-empty compact set. Let \(\hat K = K \cup \text{ bounded components 
  of } \mathbb{C} \setminus K \text{ in } U\). Then, \(\hat K\) is compact and \(\hat K \subseteq U\). 
  For every bounded component \(V\) of \(\mathbb{C} \setminus \hat K\), we know 
  \(V \setminus U \neq \varnothing\), so we can pick \(\lambda_V \in V \setminus U\). Let \(\Lambda\) 
  be the set of all such \(\lambda_v\)s. By Rangu's theorem, given \(f \in \mathcal{O}(U), \epsilon > 0\),
  there exists a rational function \(r\) whose poles lie in \(\Lambda\) and \(\|f - r\|_{\hat K} < \epsilon\).
  Thus, \(r \in \mathcal{R}(U)\) and \(\|f - r\|_K < \epsilon\) as required.
\end{proof}

\begin{proof}[Proof of HFC]
  Let \(A, x, U\) be as in the theorem. 

  \textit{Existence}: Since, we have already defined \(\Theta_x\) and shown it is unital, linear and continuous,
  we just need to check \(\Theta_x(fg) = \Theta_x(f)\Theta_x(g)\) for all \(f, g \in \mathcal{O}(U)\).
  However, we have already shown this for rational functions, and thus, we may conclude by using the density 
  of rational functions in \(\mathcal{O}(U)\) and the continuity of \(\Theta_x\). 

  \textit{Uniqueness}: If \(\Psi : \mathcal{O}(U) \to A\) is another continuous unital homomorphism such that 
  \(\Psi(u) = x\), then, as \(\Psi\) is a homomorphism, \(\Psi(r) = r(x)\) for all rational functions \(r\). 
  Thus, \(\Theta_x = \Psi\) by continuity and density.
\end{proof}

\newpage
\section{\texorpdfstring{\(C^*\)-algebra}{C*-algebra}}

\begin{definition}[\(*\)-algebra]
  A \(*\)-algebra is a complex algebra \(A\) with an involution operation, that is, a map 
  \[A \to A : x \mapsto x^*\]
  such that 
  \begin{itemize}
    \item \((\lambda x + \mu y)^* = \bar \lambda x^* + \bar \mu y^*\) for all \(\lambda, \mu \in \mathbb{C}\) and
      \(x, y \in A\),
    \item \((xy)^* = y^* x^*\) for all \(x, y \in A\),
    \item \((x^*)^* = x\) for all \(x \in A\).
  \end{itemize}
  If \(A\) is in addition unital, then \(1^* = 1\) and hence \(1^* = 1\).
\end{definition}

\begin{definition}[\(C^*\)-algebra]
  A \(C^*\)-algebra \(A\) is a Banach algebra with an involution such that the \(C^*\) equation holds: 
  \[\|x^*x\| = \|x\|^2\]
  for all \(x \in A\).

  Namely, a \(C^*\) is a \(*\)-algebra with a complete algebra norm satisfying the \(C^*\) equation. 
  Such a norm is called a \(C^*\)-norm.
\end{definition}

We remark that \(\|x^*\| = \|x\|\) since \(\|x^*\| \|x\| \ge \|x^*x\| = \|x\|^2\) implying 
\(\|x\| \le \|x^*\|\) while \(\|x\| = \|(x^*)^*\| \le \|x^*\|\). Thus, the involution operation is 
continuous and isometric.

\begin{definition}
  A Banach-\(*\)-algebra \(A\) is a Banach algebra with an involution such that for all \(x \in A\), 
  \(\|x\| = \|x^*\|\).
\end{definition}

If \(A\) is a \(C^*\)-algebra with multiplicative unit \(1 \ne 0\), then \(\|1\| = 1\) since 
\(\|1\| = \|1^* 1\| = \|1\|^2\).

\begin{definition}
  A \(*\)-subalgebra of a \(*\)-algebra \(A\) is a subalgebra which is closed under involution.
\end{definition}

\begin{definition}
  A closed \(*\)-subalgebra of a \(C^*\)-algebra is a \(C^*\)-subalgebra.
\end{definition}

Since the closure of a \(*\)-subalgebra is also a \(*\)-subalgebra, it is a \(C^*\)-subalgebra.

\begin{definition}
  A algebra homomorphism \(\theta : A \to B\) between \(*\)-algebras is a \(*\)-homomorphism 
  if for all \(x \in A\), \(\theta(x^*) = \theta(x)^*\). A \(*\)-isomorphism is a bijective 
  \(*\)-homomorphism.
\end{definition}

\begin{example}
  \(C(K)\) for \(K\) a compact Hausdorff space is a commutative unital \(C^*\)-algebra with the involution 
  \(f^*(z) = \overline{f(z)}\).
\end{example}

\begin{example}
  Given a Hilbert space \(H\), the space of bounded linear maps from \(H\) to itself \(\mathcal{B}(H)\) 
  with the involution \(T \mapsto T^*\) where \(T^*\) it the adjoint of \(T\) is a unital \(C^*\)-algebra.
\end{example}

\begin{example}
  Any \(C^*\)-subalgebra of \(\mathcal{B}(H)\) is obviously a \(C^*\)-algebra. However, it turns out 
  that any \(C^*\)-algebra is isometrically \(*\)-isomorphic to a \(C^*\)-subalgebra of \(\mathcal{B}(H)\) 
  for some Hilbert space \(H\). This is the Gelfand-Naimark theorem.
\end{example}

From now on, we will always take \(A\) to be a \(C^*\) algebra.

\begin{definition}
  For \(x \in A\), we say it is
  \begin{itemize}
    \item hermitian (or self-adjoint) if \(x^* = x\),
    \item unitary if \(A\) is unital and \(x^*x = x x^* = 1\),
    \item normal if \(x^*x = x x^*\).
  \end{itemize}
\end{definition}

\begin{example}
  If \(A\) is unital, then 1 is hermitian and unitary. In general, both hermitian and unitary elements 
  are normal.
\end{example}

In \(C(K)\), \(f \in C(K)\) is hermitian if and only if \(f(K) \subseteq \mathbb{R}\) and it is 
unital if and only if \(f(K) \subseteq \mathbb{T} = \{z \in \mathbb{C} \mid |z| = 1\}\).

For \(x \in A\), there exists unique hermitian \(h, k \in A\) such that \(x = h + ik\). Indeed, 
by observing if \(x = h + ik\) then \(x^* = h - ik\), it follows 
\[h = \frac{x + x^*}{2}, k = \frac{x - x^*}{2i}.\]
By checking, we see that this choice works. 
Furthermore, we note that \(x\) is normal if and only if \(hk = kh\).

We remark that, for \(x \in A\) unitary, \(x \in G(A)\) is and only if \(x^* \in G(A)\). So, 
\[\sigma_A(x^*) = \{\overline\lambda \mid \lambda \in \sigma_A(x)\}\]
and \(r_A(x) = r_A(x^*)\).

\begin{lemma}
 If \(x \in A\) is normal, then \(r_A(x) = \|x\|\). 
\end{lemma}
\begin{proof}
  Firstly, we consider the case that \(x\) is hermitian. In this case, 
  \(\|x^2\| = \|x^*x\| = \|x\|^2\). Thus, for all \(n \in \mathbb{N}\), \(\|x^{2n}\| = \|x\|^{2n}\) 
  and by the spectral radius formula, 
  \[r_A(x) = \lim_{n \to \infty} \|x^{2n}\|^{1 / 2n} = \|x\|\]
  as required.
  
  Now, for any \(x \in A\), \((x^* x)^* = x^* (x^*)^* = x^* x\), i.e. \(x^*x\) is hermitian, we have 
  that \(r_A(x^*x) = \|x^*x\| = \|x\|^2\). Hence, if \(x\) is normal,  
  \[\|x\|^2 = r_A(x^*x) \le r_A(x^*)r_A(x) \le r_A(x)\|x\|\]
  where \(r_A(x^*x) \le r_A(x^*)r_A(x)\) follows by corollary~\ref{cor:commut_radius}.
  Thus, \(\|x\| \le r_A(x)\). However, as we already know that \(r_A(x) \le \|x\|\), we have the 
  desired equality.
\end{proof}

\begin{lemma}
  Assume \(A\) is unital, \(x \in A\) and \(\phi \in \Phi_A\). Then \(\phi(x^*) = \overline{\phi(x)}\).
\end{lemma}
\begin{proof}
  It suffices to show that \(\phi(x) \in \mathbb{R}\) for any hermitian \(x\). Indeed, in this case, 
  for any \(x \in A\), writing \(x = h + ik\) where \(h, k \in A\) are hermitian, we have 
  \[\phi(x^*) = \phi(h - ik) = \phi(h) - i \phi(k) = \overline{\phi(h) + i\phi(k)} 
    = \overline{\phi(h + ik)} = \overline{\phi(x)}.\]
  Assume now \(x\) is hermitian and write \(\phi(x) = a + ib\) for some \(a, b \in \mathbb{R}\). Then, 
  for all \(t \in \mathbb{R}\), by recalling that \(\|\phi\| = 1\),
  \[a^2 + (b + t)^2 = |\phi(x + it1)|^2 \le \|x + it1\|^2 = \|(x + it)^*(x + it)\| = \|x^2 + t^2 1\| 
  \le \|x^2\| + t^2.\]
  Namely, \(a^2 + b^2 + 2bt \le \|x\|^2\) for all \(t\), which in turn implies \(b = 0\) as required. 
\end{proof}

The assumption that \(A\) is unital in the above is not necessary. However, in this case, unitization 
is tricky as \(A_+\) might not satisfy the \(C^*\)-equation.

\begin{corollary}
  Let \(A\) be unital, then 
  \begin{itemize}
    \item if \(x \in A\) is hermitian, then \(\sigma_A(x) \subseteq \mathbb{R}\).
    \item if \(x \in A\) is unitary, then \(\sigma_A(x) \subseteq \mathbb{T}\).
    \item if \(B \subseteq A\) is a unital \(C^*\)-subalgebra of \(A\) and \(x \in B\) is normal, then 
      \(\sigma_A(x) = \sigma_B(x)\).
  \end{itemize}
\end{corollary}
\begin{proof}\(\)\\
  \begin{itemize}
    \item If \(x\) is hermitian, then for all \(\phi \in \Phi_A\), we have \(\phi(x) \in \mathbb{R}\) 
      from above lemma. Thus, the result follows by recalling that \(\sigma_A(x) = \{\phi(x) \mid \phi \in \Phi_A\}\).
    \item If \(x\) is unitary, then \(1 = \phi(1) = \phi(x^*x) = \phi(x^*)\phi(x) = \overline{\phi(x)}\phi(x) = |\phi(x)|^2\).
    \item Firstly, let us consider the case where \(x\) is hermitian. In this case, we have by 
    theorem~\ref{thm:spectrum_subalg} that 
    \[\sigma_A(x) \subseteq \sigma_B(x) \text{ and } \partial \sigma_B(x) \subseteq \partial \sigma_A(x).\]
    However, in this case, as \(x\) is hermitian, the spectrums are subsets of \(\mathbb{R}\) and hence do 
    not have any holes. Thus, it follows \(\sigma_A(x) = \sigma_B(x)\).

    Now, let us assume that \(x \in B\) is normal. Then, taking \(\lambda \in \mathbb{C}\), we have
    \begin{align*}
      \lambda 1 - x \in G(B) & \iff \lambda 1 - x, (\lambda 1 - x)^* \in G(B)\\
      & \iff (\overline{\lambda} 1 - x^*)(\lambda 1 - x) \in G(B)\\
      & \iff (\overline{\lambda} 1 - x^*)(\lambda 1 - x) \in G(A)\\
      & \iff \lambda 1 - x \in G(A)
    \end{align*}
    where the third if and only if follows as \(x^* x\) is hermitian.

    (Stronger statement for all \(x \in B\): Murphy, Gerard J. \(C^*\)-Algebras and Operator Theory. Theorem 2.1.11)  
  \end{itemize}
\end{proof}

If \(H\) is a Hilbert space and \(T \in \mathcal{B}(H)\) is hermitian or unitary. By the above 
corollary, we have 
\[\sigma(T) = \partial \sigma(T) \subseteq \sigma_{ap}(T) \subseteq \sigma(T).\]
Hence \(\sigma(T) = \sigma_{ap}(T)\) (where \(\sigma_{ap}(T)\) is the approximate point spectrum 
of \(T\)).

\begin{theorem}[Commutative Gelfand-Naimark theorem]
  Let \(A\) be a commutative unital \(C^*\)-algebra. Then, there exists some compact Hausdorff 
  \(K\) such that \(A\) is isometrically \(*\)-isomorphic to \(C(K)\). This isomorphism is 
  precisely the Gelfand map \(x \mapsto \hat x\).
\end{theorem}
\begin{proof}
  We know the Gelfand maps is a unital homomorphism and so, it suffices to show \((\hat x)^* = \widehat{(x^*)}\),
  isometry and surjectivity.

  Let \(x \in A\), then for all \(\phi \in \Phi_A\), \((\hat x)^*(\phi) = \overline{\phi(x)} = 
  \phi(x^*) = \widehat{(x^*)}(\phi)\) as required.

  \textit{Isometric}: We already know from the Gelfand representation theorem that \(\|\hat x\|_\infty 
  = r_A(x)\). Now, since \(A\) is commutative, any \(x \in A\) is normal. Thus, it follows that 
  \(r_A(x) = \|x\|\).

  \textit{Surjectivity}: Since the Gelfand map is an isometric, unital \(*\)-homomorphism, its image 
  is a closed, unital, \(*\)-subalgebra of \(C(K)\). It furthermore separates points since for all 
  \(\phi \ne \psi \in \Phi_A\), there exists some \(x \in A\) such that \(\phi(x) \ne \psi(x)\). Thus, 
  \(\hat x(\phi) \ne \hat x(\psi)\). Hence, by Stone-Weierstrass, it follows the image must be all 
  of \(C(K)\).
\end{proof}

\begin{corollary}
  Let \(A\) be a unital \(C^*\)-algebra. \(x \in A\) is said to be positive if \(x\) is hermitian and 
  \(\sigma_A(x) \subseteq [0, \infty)\). A positive \(x \in A\) has a unique positive square root, i.e. 
  there exists a unique positive \(y \in A\) such that \(y^2 = x\).  
\end{corollary}
\begin{proof}
  \textit{Existence}: Let \(B = \langle x \rangle = \overline{\{p(x) \mid \text{polynomial } p\}}\).
  Then clearly \(x \in B\) and \(\sigma_A(x) = \sigma_B(x) \subseteq [0, \infty)\). Consider the Gelfand 
  map on \(B\) so that for all \(\phi \in \Phi_B\), \(\hat x(\phi) = \phi(x) \in \sigma_B(x) \subseteq [0, \infty)\).
  Thus, we can define \(\hat y \in C(\Phi_B)\) such that \(\hat y(\phi) = \sqrt{\hat x(\phi)}\). Then, 
  \(\hat y\) is a positive square root of \(\hat x\). Hence, by surjectivity, there exists some 
  \(y \in A\) which is a positive square root of \(x\) as required.

  \textit{Uniqueness}: Assume \(z \in A\) is another positive square root of \(x\). Then, 
  \[zx = xz = z^3.\]
  Thus, there exists a commutative \(*\)-subalgebra \(D\) of \(A\) such that \(x, z \in D\). Then, 
  by considering the Gelfand map on \(D\), we have \(\hat y = \hat z\) implying \(y = z\).
\end{proof}

This corollary applies in the case that \(A = \mathcal{B}(H)\) for some Hilbert space \(H\). 
In this case, \(T \in \mathcal{B}(H)\) is positive if and only if \(\langle Tx, x\rangle > 0\) 
for all \(x \in H\). Then, in the case that \(T\) is invertible in \(\mathcal{B}(H)\), 
there exists unique \(R, U \in \mathcal{B}(H)\)
such that \(R\) is positive, \(U\) is unitary such that \(T = RU\). Indeed, since \(TT^*\) is 
positive, we can take \(R = (TT^*)^{1 / 2}\) and \(U = R^{-1} T\). Then \(U\) is invertible 
and \(UU^* = R^{-1} TT^* R^{-1} = R^{-1} R^2 R^{-1} = 1\) implying \(T = RU\).

\end{document}


